\RequirePackage{silence}
\WarningFilter{remreset}{The remreset package}
\documentclass[11pt]{book}
\usepackage{amsmath,amsthm,amssymb}
\usepackage[mathscr]{eucal}
\usepackage[hidelinks]{hyperref}
\usepackage[margin=1in]{geometry}
\usepackage{standalone}
\usepackage{enumitem}
\usepackage{fancyhdr}
\usepackage{setspace}
\usepackage{tikz}
\usepackage{mathpazo}
\usepackage{thmtools}
\usepackage{tikz}
\usepackage{tikz-cd}
\usepackage{xcolor}
\hypersetup{
    colorlinks,
    linkcolor={red!50!black},
    citecolor={blue!50!black},
    urlcolor={blue!80!black}
}
%\usepackage{tcolorbox}

%%%%%%%%%%%%% HEADERS %%%%%%%%%%%%%%%
\pagestyle{fancy}
\fancyhf{}
\rhead{\thepage}
\lhead{\leftmark}
\headheight=13.6pt

%%%%%%%%%%%%% ENVIRONMENTS %%%%%%%%%%%%%%%
\declaretheoremstyle[headfont=\color{blue!45!black!}\normalfont\bfseries]{bluestyle}
\declaretheoremstyle[headfont=\color{green!45!black!}\normalfont\bfseries]{greenstyle}
\declaretheoremstyle[headfont=\color{red!45!black!}\normalfont\bfseries]{redstyle}
\declaretheorem[numberwithin=section, style=definition]{definition}
\declaretheorem[sharenumber=definition, style=definition]{theorem}
\declaretheorem[sharenumber=definition, style=definition]{exercise}
\declaretheorem[sharenumber=definition, style=definition]{lemma}
\declaretheorem[sharenumber=definition, style=definition]{proposition}
\declaretheorem[sharenumber=definition, style=definition]{corollary}
\declaretheorem[sharenumber=definition, style=definition]{remark}
\declaretheorem[sharenumber=definition, style=definition]{axiom}
\declaretheorem[sharenumber=definition, style=definition]{example}
\declaretheorem[sharenumber=definition, style=definition]{result}
\declaretheorem[sharenumber=definition, style=definition]{problem}
\declaretheorem[sharenumber=definition, style=definition]{conjecture}
\declaretheorem[sharenumber=definition, style=definition]{algorithm}
\declaretheorem[sharenumber=definition, style=definition]{heuristic}
\declaretheorem[sharenumber=definition, style=definition]{motivation}
\declaretheorem[sharenumber=definition, style=definition]{intuition}
\declaretheorem[sharenumber=definition, style=definition]{anecdote}
\declaretheorem[sharenumber=definition, style=definition]{abbriviation}
\declaretheorem[sharenumber=definition, style=definition]{convention}

\declaretheorem[name={Definition},sharenumber=definition, shaded={margin=0.025\linewidth, textwidth=0.95\linewidth, bgcolor={red!3!white!}}, style=redstyle]{definitionbox}

\declaretheorem[name={Theorem},sharenumber=definition, shaded={margin=0.025\linewidth, textwidth=0.95\linewidth, bgcolor={blue!3!white!}}, style=bluestyle]{theorembox}

\declaretheorem[name={Exercise},sharenumber=definition, shaded={margin=0.025\linewidth, textwidth=0.95\linewidth, bgcolor={green!3!white!}}, style=greenstyle]{exercisebox}
\newenvironment{solution}{\begin{proof}[Solution]}{\end{proof}}
\newenvironment{abc}{\begin{enumerate}[label=(\alph*)]}{\end{enumerate}}
\renewcommand{\sectionautorefname}{\S}
\renewcommand{\subsectionautorefname}{\S}
\renewcommand{\subsubsectionautorefname}{\S}


%%%%%%%%%%%%% SHORTCUTS %%%%%%%%%%%%%%%
\DeclareMathOperator{\N}{\mathbf{N}}
\DeclareMathOperator{\Z}{\mathbf{Z}}
\DeclareMathOperator{\R}{\mathbf{R}}
\DeclareMathOperator{\F}{\mathbf{F}}
\DeclareMathOperator{\Q}{\mathbf{Q}}
\DeclareMathOperator{\GL}{\mathrm{GL}}
\DeclareMathOperator{\SL}{\mathrm{SL}}
\DeclareMathOperator{\lcm}{\operatorname{lcm}}
\DeclareMathOperator{\ord}{\operatorname{ord}}
\DeclareMathOperator{\sgn}{\operatorname{sgn}}
\DeclareMathOperator{\Aut}{\operatorname{Aut}}
\DeclareMathOperator{\Inn}{\operatorname{Inn}}
\DeclareMathOperator{\End}{\operatorname{End}}
\DeclareMathOperator{\stab}{\operatorname{stab}}
\DeclareMathOperator{\orb}{\operatorname{orb}}
%\DeclareMathOperator{\ker}{\operatorname{ker}}
\DeclareMathOperator{\im}{\operatorname{im}}
\DeclareMathOperator{\IM}{\operatorname{Im}}
\DeclareMathOperator{\RE}{\operatorname{Re}}
\DeclareMathOperator{\Span}{\operatorname{span}}
\DeclareMathOperator{\spec}{\operatorname{spec}}
\DeclareMathOperator{\Char}{\operatorname{char}}
\DeclareMathOperator{\Rank}{\operatorname{rank}}
\DeclareMathOperator{\proj}{\operatorname{proj}}
\DeclareMathOperator{\normal}{\trianglelefteq}


\title{
\vspace{-2.0cm}
\Large{Project Sigma}\\
\vspace{1cm}
\huge{\bf{Algebraic Geometry}}\\
\vspace{0.4cm}
\large{Reference \& Exercise}
\vspace{3cm}}
\author{Yunhai Xiang}
\date{\today}

\begin{document}


\maketitle
\doublespacing
\tableofcontents
\singlespacing
\newpage
%\setlength{\parskip}{10pt}
\chapter{Affine Algebraic Sets}
\begin{problem}List all points in $V=\mathcal{V}(\{Y-X^2,X-Y^2\})$.
\begin{proof}Since $V=\{(x,y):y=x^2,x=y^2\}$, we have $x=y^2=(x^2)^2=x^4$ if $(x,y)\in V$. By solving $x^4-x=0$ we have that $x\in \{0,1,w,w^2\}$ where $w=e^{2\pi i/3}$. If $x=0$, then $y=0$, if $x=1$ then $y=1$. We can easily verify that $y=x^2$ and $x=y^2$ in these cases. If $x=w$ then $y=x^2=w^2$, then we can verify $x=w=w^4=y^2$. If $x=w^2$, then $y=x^2=w^4=w$, and we can verify $x=w^2=y^2$. Therefore $V=\{(0,0),(1,1),(w,w^2),(w^2,w)\}$.
\end{proof}
\end{problem}
\begin{problem}Show that $W=\{(t,t^2,t^3):t\in\mathbf C\}$ is an algebraic set.
\begin{proof}Consider $V=\mathcal{V}(\{Y-X^2,Z-X^3\})$. For $(x,y,z)\in V$, we have $y-x^2=0$ and $z-x^3=0$, so $y=x^2$ and $z=x^3$, therefore $(x,y,z)=(x,x^2,x^3)\in W$. Conversely, let $(x,y,z)=(t,t^2,t^3)\in W$, then $y-x^2=t^2-t^2=0$ and $z-x^3=t^3-t^3=0$, hence $(x,y,z)\in V$. Thus $V=W$.
\end{proof}
\end{problem}
\begin{problem}
Suppose that $C$ is an affine plane curve and $L$ is a line with $L\not\subseteq C$. Suppose that $C=\mathcal{V}(\{F\})$ where $F\in \mathbf C[X,Y]$ a polynomial of degree $n$. Show that $L\cap C$ is a finite set of no more than $n$ points. 
\begin{proof}
Suppose that $(x,y)\in L\cap C$, since $L$ is a line, we have $y=mx+c$ for some $m,c$, therefore $F(x,mx+c)=0$. We note that $\deg F(x,mx+c)\le n$ since $mx+c$ has degree $1$. By the fundamental theorem of algebra, we have $F(x,mx+c)=0$ has at most $n$ solutions. Hence $L\cap C$ is a finite set of no more than $n$ points. 
\end{proof}
\end{problem}
\begin{problem}Show that $\mathcal{V}((Y-X^2))$ is irreducible, and that $\mathcal{I}(\mathcal{V}((Y-X^2)))=(Y-X^2)$.
\begin{proof}
We will show that $(Y-X^2)$ is prime. Consider $\varphi:\mathbf C[X,Y]\rightarrow \mathbf C[X]$ given by $X\mapsto X$ and $Y\mapsto X^2$ extended to the whole ring, then $\varphi$ is a homomorphism and $\mathrm{Ker}(\varphi)=(Y-X^2)$. Hence by the first isomorphism theorem, we have $\mathbf C[X,Y]/(Y-X^2)\cong \mathbf C[X]$ is an integral domain, hence $(Y-X^2)$ is prime. Since prime ideals are radical ideals, we have $\mathcal{I}(\mathcal{V}((Y-X^2)))=(Y-X^2)$.
\end{proof}
\end{problem}
\begin{problem}
Let $I=(Y^2-X^3-X^2,X)$, then $X\in I$ since it is a generator, hence $X^2,X^3\in I$ as well. Next, $Y^2-X^3-X^2\in I$ since it is a generator, therefore, $Y^2-X^3-X^2+X^3+X^2=Y^2\in I$. Assume for sake of contradiction that $I$ is a radical ideal, then $Y\in I$ since $Y^2\in I$. Since $Y\in I$, we have $Y=U(X,Y)(Y^2-X^3-X^2)+V(X,Y)X$ for some polynomials $U(X,Y),V(X,Y)$. Let $X=0$ on both sides, then we have $Y=U(0,Y)Y^2$ as polynomials in the indeterminant $Y$. This is a contradiction since the dergee of $Y$ on the LHS is $1$ and the degree of $Y$ on the RHS is not $1$. Since $I$ is not a radical ideal, and $\mathcal{I}(D)$ must be a radical ideal, we have $\mathcal{I}(D)\ne I$.
\end{problem}

\begin{problem}
Show that $\mathcal{V}(F)\cong \mathcal{V}(G)$ where $F(X,Y)=X^2+Y^2-1$ and $G(X,Y)=X^2-Y^2-1$.
\begin{proof}
We let $\varphi:\mathcal{V}(F)\rightarrow \mathcal{V}(G)$ be $(x,y)\mapsto (x,iy)$ which is obviously a polynomial map with an inverse $\varphi^{-1}:\mathcal{V}(G)\rightarrow \mathcal{V}(F)$ given by $(x,y)\mapsto (x,-iy)$ which is also a polynomial map. We easily verify that $\varphi(\varphi^{-1}(x,y))=(x,y)$ and $\varphi^{-1}(\varphi(x,y))=(x,y)$. We note that if $(x,y)\in\mathcal{V}(F)$ then $x^2+y^2-1=0$, and we have $G(\varphi(x,y))=x^2-(iy)^2-1=x^2+y^2-1=0$. And if $(x,y)\in\mathcal{V}(G)$ then $x^2-y^2-1=0$ then $F(\varphi^{-1}(x,y))=x^2+(-iy)^2-1=x^2-y^2-1=0$. Therefore $\varphi,\varphi^{-1}$ are well-defined. Therefore $\mathcal{V}(F)\cong \mathcal{V}(G)$.
\end{proof}
\end{problem}

\begin{problem}
Let $V=\mathcal{V}(Y^2-X^3)$ and let $\phi:\mathbf{A}^1\rightarrow V$ be $\phi(t)=(t^2,t^3)$, show that $\phi$ is a bijective polynomial map which is not an isomorphism.
\begin{proof}
Assume $s\ne t$ and $(t^2,t^3)=(s^2,s^3)$ then we have $s^2=t^2$ and $s^3=t^3$. Since $s\ne t$ we have $s-t\ne 0$. Since $s^2=t^2$ we have $s^2-t^2=(s+t)(s-t)=0$. Since $s-t\ne 0$, we have $s+t=0$, thus $s=-t$, hence $s^3=(-t)^3=-t^3$. Since $s^3=t^3$ and $s^3=-t^3$, we have $t^3=-t^3$, so $t=0$. Since $t=0$ we have $s=-t=0=t$ which contradicts the hypothesis that $s\ne t$. This shows that $\phi$ is injective. 
Next, for each $(x,y)\in\mathcal{V}(\{Y^2-X^3\})$, we have $y^2-x^3=0$ and thus $y^2=x^3$. We know that $x$ has square roots $\alpha$ and $-\alpha$ for some $\alpha$. We show that one of them is also a cube root of $y$. We have $\alpha^6=(\alpha^2)^3=x^3=y^2$, therefore $y=\alpha^3$ or $y=-\alpha^3$. Since $y=\alpha^3$ or $y=(-\alpha)^3$, we have one of $\pm\alpha$ is a cube root of $y$. Let $t=\alpha$ if $\alpha$ is a cube root of $y$ and $t=-\alpha$ otherwise. We then have $\phi(t)=(t^2,t^3)=(x,y)$. Thus $\phi$ is surjective, hence bijective. Suppose for contradiction that there is a polynomial map inverse $\phi^{-1}:V\rightarrow\mathbf A^1$ which can be represented by a polynomial $f\in\mathbf C[X,Y]$. Then have $\phi^{-1}(\phi(t))=t$, so $f(t^2,t^3)=t$. We note that $[t^1]f(t^2,t^3)=0$, since for each term $aX^nY^m$, substituding $X=t^2$ and $Y=t^3$ gives $at^{2n+3m}$, and there is no $n,m$ with $2n+3m=1$. This is a contradiction since $[t^1]t=1$.
\end{proof}
\end{problem}
\begin{problem}
Let $\phi:\mathbf A^1\rightarrow V$ be $\phi(t)=(t^2-1,t(t^2-1))$ where $V=\mathcal{V}(\{Y^2-X^2(X+1)\})$. Show that $\phi$ is one-to-one and onto except at $\phi(\pm 1)=(0,0)$.
\begin{proof}Suppose that $s\ne t$ and $(s^2-1,s(s^2-1))=(t^2-1,t(t^2-1))$, we then have $s^2-1=t^2-1$ thus $s^2-t^2=(s-t)(s+t)=0$. Since $s\ne t$, we have $s=-t$. Next, since $s(s^2-1)=t(t^2-1)$ we have $-t(t^2-1)=t(t^2-1)$. Thus $t=0$ or $t^2=1$. If $t=0$ then $s=-t=0=t$ which contradicts $s\ne t$, so $t^2=1$. Thus $t=\pm 1$ and $s=\mp 1$. Thus $\phi$ is injective except at $t=\pm 1$. Next, let $(x,y)\in V$ then $y^2-x^2(x+1)=0$ so $y^2=x^2(x+1)$. Let $\alpha$ and $-\alpha$ be the square roots of $x+1$. By $y^2=x^2(x+1)$, we have $y=\alpha x$ or $y=-\alpha x$. Let $t=\alpha$ if $y=\alpha x$ and $t=-\alpha$ otherwise. We thus have $y=tx$. Since $t$ is a square root of $x+1$, we have $t^2=x+1$, so $x=t^2-1$. Thus $x=t^2-1$ and $y=t(t^2-1)$. Hence $\phi(t)=(x,y)$. Thus $\phi$ is surjective.
\end{proof}
\end{problem}
\newpage
\begin{problem}
Let $V=\mathcal{V}(\{X^2-Y^3,Y^2-Z^3\})$, and let $\overline{\alpha}:\Gamma (V)\rightarrow \mathbf C[T]$ be given by $\overline{\alpha}(X)=T^9$, $\overline{\alpha}(Y)=T^6$ and $\overline{\alpha}(Z)=T^6$. Then
\begin{enumerate}[label=(\alph*)]
	\item What is the polynomial map $f:\mathbf A^1\rightarrow V$ with $f^*=\overline{\alpha}$
	\item Show that $f$ is bijective but not an isomorphism
\end{enumerate}
\begin{proof}~\\[-1em]
\begin{enumerate}[label=(\alph*)]
	\item Define the polynomial map $f:\mathbf A^1\rightarrow V$ by $f(t)=(t^9,t^6,t^4)$ as in the proof of Theorem 1.6. We can verify that this is well-defined since $X^2-Y^3=t^{18}-t^{18}=0$ and $Y^2-Z^3=t^{12}-t^{12}=0$. We verify that the pullback $f^*(X)=[(x,y,z)\mapsto x]\circ f=T^9$, $f^*(Y)=[(x,y,z)\mapsto y]\circ f=T^6$, and $f^*(Z)=[(x,y,z)\mapsto z]\circ f=T^4$. Thus $f^*=\overline{\alpha}$.
	\item We note that $f(t)=(0,0,0)$ iff $t=0$, so we can assume $t\ne s$ are nonzero and $(t^9,t^6,t^4)=(s^9,s^6,s^4)$. Since $t^4=s^4$, we have $t\in \{ s\zeta_4, s\zeta_4^2,s \zeta_4^3\}$. Since $t^6=s^6$, we have $t\in \{ s\zeta_6, \dots, s \zeta_6^5\}$. Since $t^9=s^9$, we have $t\in \{ s\zeta_9, \dots,s\zeta_9^8\}$. Since $\gcd(9,6,4)=1$, this is a contradiction. To explain in simpler language, $t^4=s^4$ implies that the angle between $t,s$ is $90^\circ,180^\circ$ or $270^\circ$; $t^6=s^6$ implies that the angle between $t,s$ is $60^\circ,120^\circ,180^\circ,240^\circ$ or $300^\circ$; $t^9=s^9$ implies that the angle between $t,s$ is $40^\circ,80^\circ,120^\circ,160^\circ,200^\circ,240^\circ,280^\circ$ or $320^\circ$. There is no angle between $t,s$ that satisfies our requirement. Thus $f$ is injective. Next, let $(x,y,z)\in V$, we then have $x^2-y^3=0$ and $y^2-z^3=0$, thus $x^2=y^3$ and $y^2=z^3$. The $6$-th roots of $y$ are $\{\alpha,\alpha\omega,\dots,\alpha\omega^5\}$ for some $\alpha$ where $\omega=e^{\frac{2\pi i}{6}}$. Let $s$ be a $6$-th roots of $y$. Thus $s^{18}=(s^{6})^3=y^3=x^2$, so $x=\pm s^{9}$, so $x\in \{s^9,s^9\omega^3\}$. Similarly, $s^{12}=(s^6)^2=y^2=z^3$, therefore $\{z,z\omega^2,z\omega^4\}=\{s^4,s^4\omega^2,s^4\omega^4\}$, hence $z\in \{s^4,s^4\omega^2,s^4\omega^4\}$. Suppose that $x=s^9\omega^{3n}$ for $n\in\{0,1\}$ and $z=s^4\omega^{2m}$ for $m\in\{0,1,2\}$. Let $t=s\omega^{k}$ then $t$ is also a $6$-th root of unity, so $y=t^6$. Also, $x=t^9\omega^{3n-9k}$ and $z=t^4\omega^{2m-4k}$. I claim that we can always choose $k$ such that $3n\equiv 9k\pmod{6}$ and $2m\equiv 4k\pmod{6}$. Note that $3n\equiv 9k\pmod{6}$ iff $k\equiv n\pmod{2}$, and note that $2m\equiv 4k\pmod{6}$ iff $k\equiv 2m\pmod{3}$. By the Chinese remainder theorem, such $k$ can always be chosen. Hence we have $x=t^9$, $y=t^6$ and $z=t^4$. Thus $f(t)=(x,y,z)$. Thus $f$ is surjective, so $f$ is bijective. 

	We see that $f$ is not an isomorphism, since if so there is a polynomial map $g:V\rightarrow\mathbf A^1$ which can be viewed as a polynomial $g\in \mathbf C[X,Y,Z]$ which is the inverse of $f$, then by $g\circ f=\mathrm{id}$, we have $g(t^9,t^6,t^4)=t$. We note that $[t^1]g(t^9,t^6,t^4)=0$ since if $aX^pY^qZ^r$ is a term in $g(X,Y,Z)$, then substituding $X=t^9,Y=t^6,Z=t^4$ gives $at^{9p+6q+4r}$, and there is no $p,q,r$ such that $9p+6q+4r=1$. This contradicts the fact that $[t^1]t=1$.
\end{enumerate}
\end{proof}
\end{problem}
\begin{problem}
If $\phi:V\subseteq\mathbf A^n\rightarrow W \subseteq\mathbf A^m$ is an onto polynomial map, show that if $X$ is an algebraic subset of $W$ then $\phi^{-1}[X]$ is an algebraic subset of $V$, and that $X$ is irreducible if $\phi^{-1}[X]$ is irreducible.
\begin{proof}
Suppose that $X=\mathcal{V}(I)$ for some $I\subseteq \mathbf C[X_1,\dots,X_m]$, then for $x\in V$, we have
\[
x\in \phi^{-1}[X]\Longleftrightarrow \phi(x)\in X\\
\Longleftrightarrow f(\phi(x))=0,\,\forall f\in I\\
\Longleftrightarrow x\in \mathcal{V}(\{f\circ \phi:f\in I\})
\]
Therefore $\phi^{-1}[X]=\mathcal{V}(\{f\circ \phi:f\in I\})$ is algebraic. If $X=U\cup V$ where algebraic sets $U,V\subset X$ properly, then $\phi^{-1}[X]=\phi^{-1}[U]\cup\phi^{-1}[V]$. Choose $p\in X\setminus U$, and let $x$ be such that $\phi(x)=p$, then $x\in \phi^{-1}[X]\setminus \phi^{-1}[U]$, so $\phi^{-1}[U]\subset \phi^{-1}[X]$ properly, and similarly $\phi^{-1}[V]\subset \phi^{-1}[X]$ properly. Since $\phi^{-1}[U],\phi^{-1}[V]$ are algebraic as $U,V$ are algebraic, we have $\phi^{-1}[X]$ is reducible. Thus $\phi^{-1}[X]$ is irreducible implies $X$ is irreducible.
\end{proof}
\end{problem}
\newpage

\begin{problem}
Let $V\subseteq\mathbf A^n$ be a variety, show that TFAE
\begin{enumerate}[label=(\roman*)]
	\item $V$ is a point
	\item $\Gamma(V)=\mathbf C$
	\item $\dim_{\mathbf C}\Gamma(V)$ is finite
\end{enumerate}
\begin{proof}
Assume (i), then let $V=\{(x_1,\dots,x_n)\}$. We claim that $\mathcal{I}(V)=(X_1-x_1,\dots,X_n-x_n)$. Note that $\mathcal{V}((X_1-x_1,\dots,X_n-x_n))=V$ which is straightforward. Next, since $x_1,\dots,x_n\in \mathbf C$, we have
\[\mathbf C[X_1,\dots,X_n]/(X_1-x_1,\dots,X_n-x_n)\cong \mathbf C[x_1,\dots,x_n]\cong\mathbf C\]
which is an integral domain, so $(X_1-x_1,\dots,X_n-x_n)$ is prime, so it's also a radical ideal. Therefore we have $\mathcal{I}(V)=\mathcal{I}(\mathcal{V}((X_1-x_1,\dots,X_n-x_n)))=(X_1-x_1,\dots,X_n-x_n)$ by Nullstellensatz. Thus, we indeed have $\Gamma(V)=\mathbf C[X_1,\dots,X_n]/\mathcal{I}(V)=\mathbf C$. Next, assume (ii), then $\dim_{\mathbf C} \Gamma(V)=\dim_{\mathbf C}\mathbf C=1<\infty$ straightforwardly. Assume (iii), then $\Gamma(V)=\mathbf C[X_1,\dots,X_n]/\mathcal{I}(V)$ has finite dimension over $\mathbf C$. Let $i\in\{1,\dots,n\}$. We note that if $\{1,X_i,X_i^2,X_i^3,\dots\}$ is linearly independent then we cannot have $\dim_{\mathbf C}\Gamma(V)<\infty$, thus they are linearly dependent. This means that there exists some polynomial $f_i\in\mathbf C[X_i]\subseteq \mathbf C[X_1,\dots,X_n]$ with coefficients not all zero for which $f_i(X_i)\equiv 0 \pmod{\mathcal{I}(V)}$. Hence $f_i\in\mathcal{I}(V)$ for each $i$. By Hilbert's Nullstellensatz, we have $\mathcal{V}(\mathcal{I}(V))=V$ as $V$ is an algebraic set. Thus for each $p\in V$, we have $p\in\mathcal{V}(\mathcal{I}(V))$, so $f_i(p)=0$ for each $i$. The fact that each $f_i$ is a single-variable polynomial over $\mathbf C$ means that it has finitely many roots. Therefore we only have finitely many choices for each coordinate of $p$. Thus $V$ is a finite set. Since $V$ is a variety, it is irreducible, therefore it must be a single point. 
\end{proof}
\end{problem}
\begin{problem}
Decompose $\mathcal{V}(Y^4-X^2,Y^4-X^2Y^2+XY^2-X^3)$ into irreducible components. 
\begin{proof}
We note that $Y^4-X^2=(Y^2-X)(Y^2+X)$ and $Y^4-X^2Y^2+XY^2-X^3=(X+Y)(Y-X)(X+Y^2)$. We note that $X+Y^2$ and $Y^2-X$ are irreducible. Let $V=\mathcal{V}(Y^4-X^2,Y^4-X^2Y^2+XY^2-X^3)$ then
\[\begin{aligned}V&=\mathcal{V}(Y^4-X^2)\cap \mathcal{V}(Y^4-X^2Y^2+XY^2-X^3)\\
&=(\mathcal{V}(Y^2-X)\cup\mathcal{V}(Y^2+X))\cap (\mathcal{V}(X+Y)\cup\mathcal{V}(Y-X)\cup\mathcal{V}(X+Y^2))\\
&=(\mathcal{V}(Y^2-X)\cap (\mathcal{V}(X+Y)\cup\mathcal{V}(Y-X)\cup\mathcal{V}(X+Y^2)))\cup \mathcal{V}(Y^2+X)
\end{aligned}
\]
We note that if $(x,y)\in \mathcal{V}(Y^2-X)\cap \mathcal{V}(X+Y)$ then $x+y=0$ and $y^2-x=0$ so $y^2+y=0$, which implies that $(x,y)\in\{(0,0),(1,-1)\}$. If $(x,y)\in \mathcal{V}(Y^2-X)\cap \mathcal{V}(Y-X)$ then $y-x=0$ and $y^2-x=0$, which implies $x^2-x=0$ so $(x,y)\in \{(1,1),(0,0)\}$. If $(x,y)\in \mathcal{V}(Y^2-X)\cap\mathcal{V}(X+Y^2)$ then $y^2-x=0$ and $x+y^2=0$ so $2y^2=0$ so $(x,y)=(0,0)$. Therefore
\[\begin{aligned}V&=\{(0,0),(1,-1)\}\cup\{(1,1),(0,0)\}\cup \{(0,0)\}\cup\mathcal{V}(Y^2+X)\\
&=\mathcal{V}(X,Y)\cup\mathcal{V}(X-1,Y+1)\cup\mathcal{V}(X-1,Y-1)\cup\mathcal{V}(Y^2+X)
\end{aligned}\]
The first three components are irreducible since they are single points. The last component $\mathcal{V}(Y^2+X)$ is irreducible since $(Y^2+X)$ is prime, and hence also radical, so $\mathcal{I}(\mathcal{V}(Y^2+X))=(Y^2+X)$ which is prime. %To see that $(Y^2+X)$ is prime, we note that $\varphi:\mathbf C[X,Y]\rightarrow \mathbf C[T]$ by $X\mapsto -T^2$ and $Y\mapsto T$ induces an isomorphism $\mathbf C[X,Y]/(Y^2+X)\cong \mathbf C[X,Y]/\mathrm{Ker}(\varphi)\cong \mathrm{Im}(\varphi)\cong\mathbf C[T]$ which is an integral domain. 
We note that $(Y^2+X)$ is prime since $Y^2+X$ is prime, and $Y^2+X$ is prime since $Y^2+X$ is irreducible and $\mathbf C[X,Y]$ is a ufd.
\end{proof}
\end{problem}
\newpage
\begin{problem}
Find all irreducible components of $\mathcal{V}(2X^3-X^2Y-2XY+Y^2)$.
\begin{proof}
We note that if $(x,y)\in\mathcal{V}(2X^3-X^2Y-2XY+Y^2)$, then $2x^3-(x^2+2x)y+y^2=0$, so $y=\frac{x^2+2x\pm\sqrt{(x^2+2x)^2-8x^3}}{2}=\frac{x(x+2)\pm x(x-2)}{2}$. Therefore $y=x^2$ or $y=2x$. Conversely, if $y=2x$, then $2x^3-x^2y-2xy+y^2=2x^3-2x^3-4x^2+4x^2=0$. If $y=x^2$ then $2x^3-x^2y-2xy+y^2=2x^3-x^4-2x^3+x^4=0$. Therefore we have $\mathcal{V}(2X^3-X^2Y-2XY+Y^2)=\mathcal{V}(Y-2X)\cup\mathcal{V}(Y-X^2)$. We note that $\mathcal{V}(Y-X^2)$ is irreducible by a previous assignment. Also $\mathcal{V}(Y-2X)$ is irreducible since $(Y-2X)$ is prime and hence also radical so by Nullstellensatz we have $\mathcal{I}(\mathcal{V}(Y-2X))=(Y-2X)$ which is prime. %To see that $(Y-2X)$ is prime, we note that $\varphi:\mathbf C[X,Y]\rightarrow \mathbf C[T]$ by $X\mapsto T$ and $Y\mapsto 2T$ induces an isomorphism $\mathbf C[X,Y]/(Y-2X)\cong \mathbf C[X,Y]/\mathrm{Ker}(\varphi)\cong \mathrm{Im}(\varphi)\cong\mathbf C[T]$ which is an integral domain.
We note that $(Y-2X)$ is prime since $Y-2X$ is prime, and $Y-2X$ is prime since $Y-2X$ is irreducible and $\mathbf C[X,Y]$ is a ufd.
\end{proof}
\end{problem}
\begin{problem}
Let $V=\mathcal{V}(Y^2-X^2(X+1))$ and $z=Y/X\in K (V)$, find the pole sets of $z$ and $z^2$.
\begin{proof}
First, we note that $z=Y/X=X(X+1)/Y$ since $Y^2\equiv X^2(X+1)\pmod{\mathcal{I}(V)}$. Thus if $x\ne 0$ then the expression $z=Y/X$ is defined, and if $y\ne 0$ then the expression $z=X(X+1)/Y$ is defined. Thus $z$ is defined for all $(x,y)\ne (0,0)$. For $(x,y)=(0,0)$, suppose that $z$ is defined then exists expression $r(X,Y)/s(X,Y)=X/Y$ with $s(0,0)\ne 0$. Since $s(0,0)\ne 0$, we know that $s(X,Y)$ has a nonzero constant term. Next, $Yr(X,Y)\equiv Xs(X,Y)\pmod{\mathcal I(V)}$ so $Yr(X,Y)- Xs(X,Y)=h(X,Y)(Y^2-X^2(X+1))$. Consider the coefficient of $X=X^1Y^0$ on both sides. For the LHS, since $s(X,Y)$ has a nonzero constant term, the coefficient of $X=X^1Y^0$ in $-Xs(X,Y)$ is $-1$. Since $Yr(X,Y)$ has $Y$ as a factor, the coefficient of $X=X^1Y^0$ in $Yr(X,Y)$ is $0$. Next, for the RHS. Since $Y^2h(X,Y)$ has $Y$ as a factor, coefficient of $X=X^1Y^0$ in $Y^2h(X,Y)$ is $0$. Since $-X^2(X+1)h(X,Y)=-X^3h(X,Y)-X^2h(X,Y)$, we have the coefficient of $X=X^1Y^0$ in $-X^2(X+1)h(X,Y)$ is $0$. Thus the coefficient of $X=X^1Y^0$ in the LHS is $-1$ but it is $0$ in the RHS, contradiction. Next, we note that $z^2=Y^2/X^2=(X+1)$ since $Y^2\equiv X^2(X+1)\pmod{\mathcal{I}(V)}$. The denominator of $X+1$ is $1$, so $z^2$ is defined on all points in $V$. Hence $z$ has pole at $(0,0)$ only and $z^2$ has no pole.
\end{proof}
\end{problem}
\begin{problem}
Let $F(X,Y)=Y^2-X^3+X$, $W=\mathcal{V}(F)$ and $P=(0,0)$.
\begin{enumerate}[label=(\alph*)]
	\item Show that $aX+bY$ is an element of the maximal ideal $\mathcal{M}_P(W)$ of the local ring $\mathcal{O}_P(W)$,
	\item Show that $aX+bY$ is an element of $\mathcal{M}_P(W)^2$ iff $aX+bY$ is tangent to $W$ at $P$.
\end{enumerate}
\begin{proof}~\\[-1em]
\begin{enumerate}[label=(\alph*)]
\item We recall that
\[\mathcal{M}_{P}(W)=\left\{f \in K(W) \mid f=\frac{r}{s} \text { for } r(P)=0, s(P) \neq 0\right\}\]
and in this case where $P=(0,0)$, the denominator of $aX+bY$ is always $1$ which is nonzero and when $(x,y)=(0,0)$ the numerator is $a0+b0=0$. Thus $aX+bY\in \mathcal{M}_{P}(W)$. 
\item
We note that the line $aX+bY$ is tangent to $W$ at $P$ iff their intersection $\mathcal{V}(aX+bY)\cap W=\{P\}$. We also note that $\mathcal{M}_P(W)=\mathcal{I}(P)\mathcal{O}_P(W)$. Since $\mathcal{I}(P)=(X,Y)$, we have $\mathcal{I}(P)^2=(X^2,XY,YX,Y^2)=(X^2,XY,Y^2)$. We note that $\mathcal{I}(P)=\mathcal{I}(\mathcal{V}(aX+bY))\cup\mathcal{I}(W)$.
\end{enumerate}
\end{proof}
\end{problem}
\begin{problem}Let $W=\mathcal{V}(Y^2-X^2(X+1))$, and $P=(0,0)$, show that for each $a,b\in\mathbf C$, $aX+bY$ not an element of $\mathcal{M}_P(W)^2$ unless $a=b=0$.
\end{problem}
\begin{problem}Let $C=\mathcal{V}(Y^2-X^3)$, show that the function field $K(C)$ of $C$ is isomorphic to $\mathbf C(T)$ but $\Gamma(C)$ is not isomorphic to $\mathbf C[T]$.
\begin{proof}
First, we claim that $(Y^2-X^3)$ is a prime ideal in $\mathbf C[X,Y]$. We claim that $Y^2-X^3$ is irreducible. We note that if $Y^2-X^3=f(X,Y)g(X,Y)$ then the degree of $Y$ in $f(X,Y),g(X,Y)$ are $1$. If $f(X,Y)=f_0(X)+f_1(X)Y$ and $g(X,Y)=g_0(X)+g_1(X)Y$ then $f(X,Y)g(X,Y)=f_0(X)g_0(X)+(f_0(X)g_1(X)+g_0(X)f_1(X))Y+f_1(X)g_1(X)Y^2$. Thus $f_0(X)g_0(X)=-X^3$ and we have $f_0(X)g_1(X)=g_0(X)f_1(X)$ and $f_1(X)g_1(X)=1$. By $f_1(X)g_1(X)=1$, we have $f_1,g_1$ are nonzero constants, thus by $f_0(X)g_1(X)=g_0(X)f_1(X)$, we have $f_0(X)=cg_0(X)$ for a nonzero constant $c$. Thus $f_0(X)g_0(X)=cg_0(X)^2$ which has a even degree, but $f_0(X)g_0(X)=-X^3$ which has odd degree, contradiction. Since $Y^2-X^3$ is irreducible, it is a prime as $\mathbf C[X,Y]$ is a ufd. Thus $(Y^2-X^3)$ is a prime ideal. Hence $\Gamma(C)=\mathbf C[X,Y]/(Y^2-X^3)$ by Nullstellensatz. We claim that $\Gamma(C)$ is not isomorphic to $\mathbf C[T]$. We note that $\mathbf C[T]$ is integrally closed since it is a ufd. However $\Gamma(C)\cong \mathbf C[\alpha^2,\alpha^3]$ a trancedental extension of $\mathbf C$ where $\alpha$ is not a root of any polynomial. Therefore $\Gamma(C)$ is not integrally closed, so it is not isomorphic to $\mathbf C[T]$. On the other hand, the field of fraction of $\mathbf C[\alpha^2,\alpha^3]$, which is $K(C)$, is isomorphic to $\mathbf C(T)$ as we can set $T=\alpha=\alpha^3/\alpha^2$
\end{proof}
\end{problem}
\begin{problem}Let $V=\mathcal{V}(Y-X^2)$ and $P=(1,1)$, which of the three rational functions are equal in $\mathcal{O}_P(V)$?
\begin{enumerate}
	\item $F_1(X,Y)=\frac{1}{X+1}$,
	\item $F_2(X,Y)=\frac{X}{X+Y}$,
	\item $F_3(X,Y)=\frac{X^2}{X+Y^2}$.
\end{enumerate}
\begin{proof}

\end{proof}
\end{problem}
\end{document}
