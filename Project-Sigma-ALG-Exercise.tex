\RequirePackage{silence}
\WarningFilter{remreset}{The remreset package}
\documentclass[11pt]{book}
\usepackage{amsmath,amsthm,amssymb}
\usepackage[mathscr]{eucal}
\usepackage[hidelinks]{hyperref}
\usepackage[margin=1in]{geometry}
\usepackage{standalone}
\usepackage{enumitem}
\usepackage{fancyhdr}
\usepackage{setspace}
\usepackage{tikz}
\usepackage{mathpazo}
\usepackage{thmtools}
\usepackage{tikz}
\usepackage{tikz-cd}
\usepackage{xcolor}
\hypersetup{
    colorlinks,
    linkcolor={red!50!black},
    citecolor={blue!50!black},
    urlcolor={blue!80!black}
}
%\usepackage{tcolorbox}

%%%%%%%%%%%%% HEADERS %%%%%%%%%%%%%%%
\pagestyle{fancy}
\fancyhf{}
\rhead{\thepage}
\lhead{\leftmark}
\headheight=13.6pt

%%%%%%%%%%%%% ENVIRONMENTS %%%%%%%%%%%%%%%
\declaretheoremstyle[headfont=\color{blue!45!black!}\normalfont\bfseries]{bluestyle}
\declaretheoremstyle[headfont=\color{green!45!black!}\normalfont\bfseries]{greenstyle}
\declaretheoremstyle[headfont=\color{red!45!black!}\normalfont\bfseries]{redstyle}
\declaretheorem[numberwithin=section, style=definition]{definition}
\declaretheorem[sharenumber=definition, style=definition]{theorem}
\declaretheorem[sharenumber=definition, style=definition]{exercise}
\declaretheorem[sharenumber=definition, style=definition]{lemma}
\declaretheorem[sharenumber=definition, style=definition]{proposition}
\declaretheorem[sharenumber=definition, style=definition]{corollary}
\declaretheorem[sharenumber=definition, style=definition]{remark}
\declaretheorem[sharenumber=definition, style=definition]{axiom}
\declaretheorem[sharenumber=definition, style=definition]{example}
\declaretheorem[sharenumber=definition, style=definition]{result}
\declaretheorem[sharenumber=definition, style=definition]{problem}
\declaretheorem[sharenumber=definition, style=definition]{conjecture}
\declaretheorem[sharenumber=definition, style=definition]{algorithm}
\declaretheorem[sharenumber=definition, style=definition]{heuristic}
\declaretheorem[sharenumber=definition, style=definition]{motivation}
\declaretheorem[sharenumber=definition, style=definition]{intuition}
\declaretheorem[sharenumber=definition, style=definition]{anecdote}
\declaretheorem[sharenumber=definition, style=definition]{abbriviation}
\declaretheorem[sharenumber=definition, style=definition]{convention}

\declaretheorem[name={Definition},sharenumber=definition, shaded={margin=0.025\linewidth, textwidth=0.95\linewidth, bgcolor={red!3!white!}}, style=redstyle]{definitionbox}

\declaretheorem[name={Theorem},sharenumber=definition, shaded={margin=0.025\linewidth, textwidth=0.95\linewidth, bgcolor={blue!3!white!}}, style=bluestyle]{theorembox}

\declaretheorem[name={Exercise},sharenumber=definition, shaded={margin=0.025\linewidth, textwidth=0.95\linewidth, bgcolor={green!3!white!}}, style=greenstyle]{exercisebox}
\newenvironment{solution}{\begin{proof}[Solution]}{\end{proof}}
\newenvironment{abc}{\begin{enumerate}[label=(\alph*)]}{\end{enumerate}}
\renewcommand{\sectionautorefname}{\S}
\renewcommand{\subsectionautorefname}{\S}
\renewcommand{\subsubsectionautorefname}{\S}


%%%%%%%%%%%%% SHORTCUTS %%%%%%%%%%%%%%%
\DeclareMathOperator{\N}{\mathbf{N}}
\DeclareMathOperator{\Z}{\mathbf{Z}}
\DeclareMathOperator{\R}{\mathbf{R}}
\DeclareMathOperator{\F}{\mathbf{F}}
\DeclareMathOperator{\Q}{\mathbf{Q}}
\DeclareMathOperator{\GL}{\mathrm{GL}}
\DeclareMathOperator{\SL}{\mathrm{SL}}
\DeclareMathOperator{\lcm}{\operatorname{lcm}}
\DeclareMathOperator{\ord}{\operatorname{ord}}
\DeclareMathOperator{\sgn}{\operatorname{sgn}}
\DeclareMathOperator{\Aut}{\operatorname{Aut}}
\DeclareMathOperator{\Inn}{\operatorname{Inn}}
\DeclareMathOperator{\End}{\operatorname{End}}
\DeclareMathOperator{\stab}{\operatorname{stab}}
\DeclareMathOperator{\orb}{\operatorname{orb}}
%\DeclareMathOperator{\ker}{\operatorname{ker}}
\DeclareMathOperator{\im}{\operatorname{im}}
\DeclareMathOperator{\IM}{\operatorname{Im}}
\DeclareMathOperator{\RE}{\operatorname{Re}}
\DeclareMathOperator{\Span}{\operatorname{span}}
\DeclareMathOperator{\spec}{\operatorname{spec}}
\DeclareMathOperator{\Char}{\operatorname{char}}
\DeclareMathOperator{\Rank}{\operatorname{rank}}
\DeclareMathOperator{\proj}{\operatorname{proj}}
\DeclareMathOperator{\normal}{\trianglelefteq}


\title{
\vspace{-2.0cm}
\Large{Project Sigma}\\
\vspace{1cm}
\huge{\bf{Algebraic Geometry}}\\
\vspace{0.4cm}
\large{Reference \& Exercise}
\vspace{3cm}}
\author{Yunhai Xiang}
\date{\today}

\begin{document}


\maketitle
\doublespacing
\tableofcontents
\singlespacing
\newpage
%\setlength{\parskip}{10pt}
\chapter{Affine Algebraic Sets}
\begin{problem}List all points in $V=\mathcal{V}(\{Y-X^2,X-Y^2\})$.
\begin{proof}Since $V=\{(x,y):y=x^2,x=y^2\}$, we have $x=y^2=(x^2)^2=x^4$ if $(x,y)\in V$. By solving $x^4-x=0$ we have that $x\in \{0,1,w,w^2\}$ where $w=e^{2\pi i/3}$. If $x=0$, then $y=0$, if $x=1$ then $y=1$. We can easily verify that $y=x^2$ and $x=y^2$ in these cases. If $x=w$ then $y=x^2=w^2$, then $x=w=w^4=y^2$. If $x=w^2$, then $y=x^2=w^4=w$, and $x=w^2=y^2$. Therefore 
\[V=\{(0,0),(1,1),(w,w^2),(w^2,w)\}\]
\end{proof}
\end{problem}
\begin{problem}Show that $W=\{(t,t^2,t^3):t\in\mathbf C\}$ is an algebraic set.
\begin{proof}Consider $V=\mathcal{V}(\{Y-X^2,Z-X^3\})$. For $(x,y,z)\in V$, we have $y=x^2$ and $z=x^3$, therefore $(x,y,z)=(x,x^2,x^3)\in W$. Conversely, let $(x,y,z)=(t,t^2,t^3)\in W$, then $y-x^2=t^2-t^2=0$ and $z-x^3=t^3-t^3=0$, hence $(x,y,z)\in V$. Thus $V=W$.
\end{proof}
\end{problem}
\begin{problem}
Suppose that $C$ is an affine plane curve and $L$ is a line with $L\not\subseteq C$. Suppose that $C=\mathcal{V}(\{F\})$ where $F\in \mathbf C[X,Y]$ a polynomial of degree $n$. Show that $L\cap C$ is a finite set of no more than $n$ points. 
\begin{proof}
Suppose that $(x,y)\in L\cap C$, since $L$ is a line, we have $y=mx+c$ for some $m,c$, therefore $F(x,mx+c)=0$. We note that $\deg F(x,mx+c)\le n$ since $mx+c$ has degree $1$. By the fundamental theorem of algebra, we have $F(x,mx+c)=0$ has at most $n$ solutions. Hence $L\cap C$ is a finite set of no more than $n$ points. 
\end{proof}
\end{problem}
\begin{problem}Show that $\mathcal{V}((Y-X^2))$ is irreducible, and that $\mathcal{I}(\mathcal{V}((Y-X^2)))=(Y-X^2)$.
\begin{proof}
We will show that $(Y-X^2)$ is prime. Consider $\varphi:\mathbf C[X,Y]\rightarrow \mathbf C[X]$ given by $X\mapsto X$ and $Y\mapsto X^2$ extended to the whole ring, then $\varphi$ is a homomorphism and $\mathrm{Ker}(\varphi)=(Y-X^2)$. Hence by the first isomorphism theorem, we have $\mathbf C[X,Y]/(Y-X^2)\cong \mathbf C[X]$ is an integral domain, hence $(Y-X^2)$ is prime. Since prime ideals are radical ideals, we have $\mathcal{I}(\mathcal{V}((Y-X^2)))=(Y-X^2)$
\end{proof}
\end{problem}
\end{document}
