\RequirePackage{silence}
\WarningFilter{remreset}{The remreset package}
\documentclass[11pt]{book}
\usepackage{amsmath,amsthm,amssymb}
\usepackage[mathscr]{eucal}
\usepackage[hidelinks]{hyperref}
\usepackage[margin=1in]{geometry}
\usepackage{standalone}
\usepackage{enumitem}
\usepackage{fancyhdr}
\usepackage{setspace}
\usepackage{tikz}
\usepackage{mathpazo}
\usepackage{thmtools}
\usepackage{tikz}
\usepackage{tikz-cd}
\usepackage{xcolor}
\hypersetup{
    colorlinks,
    linkcolor={red!50!black},
    citecolor={blue!50!black},
    urlcolor={blue!80!black}
}
%\usepackage{tcolorbox}

%%%%%%%%%%%%% HEADERS %%%%%%%%%%%%%%%
\pagestyle{fancy}
\fancyhf{}
\rhead{\thepage}
\lhead{\leftmark}
\headheight=13.6pt

%%%%%%%%%%%%% ENVIRONMENTS %%%%%%%%%%%%%%%
\declaretheoremstyle[headfont=\color{blue!45!black!}\normalfont\bfseries]{bluestyle}
\declaretheoremstyle[headfont=\color{green!45!black!}\normalfont\bfseries]{greenstyle}
\declaretheoremstyle[headfont=\color{red!45!black!}\normalfont\bfseries]{redstyle}
\declaretheorem[numberwithin=section, style=definition]{definition}
\declaretheorem[sharenumber=definition, style=definition]{theorem}
\declaretheorem[sharenumber=definition, style=definition]{exercise}
\declaretheorem[sharenumber=definition, style=definition]{lemma}
\declaretheorem[sharenumber=definition, style=definition]{proposition}
\declaretheorem[sharenumber=definition, style=definition]{corollary}
\declaretheorem[sharenumber=definition, style=definition]{remark}
\declaretheorem[sharenumber=definition, style=definition]{axiom}
\declaretheorem[sharenumber=definition, style=definition]{example}
\declaretheorem[sharenumber=definition, style=definition]{result}
\declaretheorem[sharenumber=definition, style=definition]{problem}
\declaretheorem[sharenumber=definition, style=definition]{conjecture}
\declaretheorem[sharenumber=definition, style=definition]{algorithm}
\declaretheorem[sharenumber=definition, style=definition]{heuristic}
\declaretheorem[sharenumber=definition, style=definition]{motivation}
\declaretheorem[sharenumber=definition, style=definition]{intuition}
\declaretheorem[sharenumber=definition, style=definition]{anecdote}
\declaretheorem[sharenumber=definition, style=definition]{abbriviation}
\declaretheorem[sharenumber=definition, style=definition]{convention}

\declaretheorem[name={Definition},sharenumber=definition, shaded={margin=0.025\linewidth, textwidth=0.95\linewidth, bgcolor={red!3!white!}}, style=redstyle]{definitionbox}

\declaretheorem[name={Theorem},sharenumber=definition, shaded={margin=0.025\linewidth, textwidth=0.95\linewidth, bgcolor={blue!3!white!}}, style=bluestyle]{theorembox}

\declaretheorem[name={Exercise},sharenumber=definition, shaded={margin=0.025\linewidth, textwidth=0.95\linewidth, bgcolor={green!3!white!}}, style=greenstyle]{exercisebox}
\newenvironment{solution}{\begin{proof}[Solution]}{\end{proof}}
\newenvironment{abc}{\begin{enumerate}[label=(\alph*)]}{\end{enumerate}}
\renewcommand{\sectionautorefname}{\S}
\renewcommand{\subsectionautorefname}{\S}
\renewcommand{\subsubsectionautorefname}{\S}


%%%%%%%%%%%%% SHORTCUTS %%%%%%%%%%%%%%%
\DeclareMathOperator{\N}{\mathbf{N}}
\DeclareMathOperator{\Z}{\mathbf{Z}}
\DeclareMathOperator{\R}{\mathbf{R}}
\DeclareMathOperator{\F}{\mathbf{F}}
\DeclareMathOperator{\Q}{\mathbf{Q}}
\DeclareMathOperator{\GL}{\mathrm{GL}}
\DeclareMathOperator{\SL}{\mathrm{SL}}
\DeclareMathOperator{\lcm}{\operatorname{lcm}}
\DeclareMathOperator{\ord}{\operatorname{ord}}
\DeclareMathOperator{\sgn}{\operatorname{sgn}}
\DeclareMathOperator{\Aut}{\operatorname{Aut}}
\DeclareMathOperator{\Inn}{\operatorname{Inn}}
\DeclareMathOperator{\End}{\operatorname{End}}
\DeclareMathOperator{\stab}{\operatorname{stab}}
\DeclareMathOperator{\orb}{\operatorname{orb}}
%\DeclareMathOperator{\ker}{\operatorname{ker}}
\DeclareMathOperator{\im}{\operatorname{im}}
\DeclareMathOperator{\IM}{\operatorname{Im}}
\DeclareMathOperator{\RE}{\operatorname{Re}}
\DeclareMathOperator{\Span}{\operatorname{span}}
\DeclareMathOperator{\spec}{\operatorname{spec}}
\DeclareMathOperator{\Char}{\operatorname{char}}
\DeclareMathOperator{\Rank}{\operatorname{rank}}
\DeclareMathOperator{\proj}{\operatorname{proj}}
\DeclareMathOperator{\normal}{\trianglelefteq}


\title{
\vspace{-2.0cm}
\Large{Project Sigma}\\
\vspace{1cm}
\huge{\bf{Algebraic Geometry}}
\vspace{3cm}}
\author{Yunhai Xiang}
\date{\today}

\begin{document}


\maketitle
\doublespacing
\tableofcontents
\singlespacing
\newpage
%\setlength{\parskip}{10pt}
\chapter{Affine Algebraic Sets}
Let $k$ be a field and $n\in\mathbf N$, then the \textit{affine space} $\mathbf A^n(k)$ of dimension $n$, or simply $\mathbf A^n$ if it does not cause confusion, is the same structure as the $n$-dimensional vector space $k^n$ over $k$, except with affine maps as morphisms, where an affine map is a linear map shifted by a constant. In this course note, unless otherwise specified, I will take $k=\mathbf C$ the set of complex numbers.

\begin{definition}
Suppose that $S\subseteq k[X_1,\dots,X_n]$ for some $n\in\mathbf N$, we define
\[\mathcal{V}(S)=\{x\in \mathbf A^n(k):\forall f\in S,\,f(x)=0\}\]
as the \textit{zero-locus} of $S$. A subset of $\mathbf A^n(k)$ that is the zero-locus of some $S$ is called \textit{(affine) algebraic}.
\end{definition}
For example, in $\mathbf{A}^2(\mathbf R)$, the sets $\mathcal{V}(\{Y\})$ and $\mathcal{V}(\{X\})$ are the $x$-axis and the $y$-axis, the set $\mathcal{V}(\{X,Y\})$ is the origin. The set $\mathcal{V}(\{f\})$, where $f\in k[X,Y]$ is polynomial of degree $2$, is known as a \textit{conic section}, for example, the circle $\mathcal{V}(\{X^2+Y^2-1\})$, the parabola $\mathcal{V}(\{Y-X^2\})$, and the hyperbola $\mathcal{V}(\{XY-1\})$. These are all examples of algebraic sets. We should also mention examples of non-algebraic sets. The set $\mathbf Z$ considered as a subset of $\mathbf A^1(\mathbf R)$ is obviously not algebraic. Next, we claim that to find all algebraic sets, we need not consider all subsets of $k[X_1,\dots,X_n]$. Let $R$ be a commutative ring, we recall the following theorems.
\begin{theorem}$R$ is noetherian iff all ideals $I\subseteq R$ are finitely generated.
\end{theorem}
\begin{theorem}[Hilbert's Basis theorem]If $R$ is noetherian, then so is $R[X_1,\dots,X_n]$.
\end{theorem}

Therefore $k[X_1,\dots,X_n]$ is noetherian, and hence every for each $S\subseteq k[X_1,\dots,X_n]$, the ideal $\langle S\rangle$ is generated by some $f_1,\dots,f_m\in k[X_1,\dots,X_n]$. We claim that $\mathcal{V}(S)=\mathcal{V}(\langle S\rangle)=\mathcal{V}(\{f_1,\dots,f_m\})$. First, we know that $\mathcal{V}(\langle S\rangle)\subseteq \mathcal{V}(S)$ since $S\subseteq\langle S\rangle$. Conversely, suppose that $f\in\langle S\rangle$, then there exists $g_1,\dots,g_\ell\in S$ and $\lambda_1,\dots,\lambda_\ell\in k[X_1,\dots,X_n]$ with $f=\lambda_1g_1+\cdots+\lambda_{\ell}g_{\ell}$. Suppose that $p\in \mathcal{V}(S)$, then $f(p)=\lambda_1g_1(p)+\cdots+\lambda_{\ell}g_{\ell}(p)=0$. Since all $p\in \mathcal{V}(S)$ is a zero of all $f\in \langle S\rangle$, we must have $\mathcal{V}(S)\subseteq \mathcal{V}(\langle S\rangle)$, and hence $\mathcal{V}(S)=\mathcal{V}(\langle S\rangle)$. The equality $\mathcal{V}(\langle S\rangle)=\mathcal{V}(\{f_1,\dots,f_m\})$ is derived similarly. Hence all algebraic sets are the zero loci of ideals, and also all algebraic sets are the zero loci of finite sets. Conversely, we can define an ideal $\mathcal{I}(X)$ for each $X\subseteq \mathbf A^n(k)$. 

\begin{definition}Let $X\subseteq \mathbf A^n(k)$, then define the ideal $\mathcal{I}(X)$ of $k[X_1,\dots,X_n]$ as 
\[\mathcal{I}(X)=\{f\in k[X_1,\dots,X_n]:\forall p\in X,\,f(p)=0\}\]
which is a well-defined ideal as we can verify easily.
\end{definition}
In fact, not only is $\mathcal{I}(X)$ an ideal, it is also a radical ideal. We recall that an radical ideal of a commutative ring $R$ is an ideal $I\subseteq R$ with $I=\sqrt{I}$ where $\sqrt{I}=\{r\in R:\exists m>0,\, r^m\in I\}$. In other words, a radical ideal is an ideal $I\subseteq R$ where for all $r\in R$, if $r^m\in I$ for some $m>0$, then $r\in I$. To see that $\mathcal{I}(X)$ is a radical ideal, note that if $f^m\in \mathcal{I}(X)$ for some $m>0$, then $f^m(p)=0$ for all $p\in X$, then $f(p)=0$ for all $p\in X$ as $k$ is a field, hence $f\in \mathcal{I}(X)$. 

Similar to how $\mathcal{V}(S)\subseteq \mathcal{V}(T)$ when $T\subseteq S\subseteq k[X_1,\dots,X_n]$, we easily have $\mathcal{I}(X)\subseteq \mathcal{I}(Y)$ when $Y\subseteq X\subseteq \mathbf A^n(k)$. Let $f\in S\subseteq k[X_1,\dots,X_n]$, then by definition $f$ vanishes on all of $\mathcal{V}(S)$, therefore $f\in \mathcal{I}(\mathcal{V}(S))$. Let $p\in X\subseteq \mathbf A^n(k)$, then by definition $p$ is a zero of all polynomials of $\mathcal{I}(X)$, therefore $p\in \mathcal{V}(\mathcal{I}(X))$. Hence we have $S\subseteq \mathcal{I}(\mathcal{V}(S))$ and $X\subseteq \mathcal{V}(\mathcal{I}(X))$. From these facts, we derive that $\mathcal{I}(X)= \mathcal{I}(\mathcal{V}(\mathcal{I}(X)))$ and $\mathcal{V}(S)= \mathcal{V}(\mathcal{I}(\mathcal V(S)))$. In fact, we have the following.
\begin{theorembox}There is a bijective correspondance
\[\{\textrm{radical\ ideals\ of\ }k[X_1,\dots,X_n]\}\longleftrightarrow\{\textrm{algebraic\ sets\ of\ }\mathbf A^n(k)\}\]
given by $I\mapsto \mathcal{V}(I)$ and $X\mapsto \mathcal{I}(X)$.
\end{theorembox}
whose proof we will delay until later in this note. This observation is central to algebraic geometry. 

We observe that for a nonempty family of ideals $I_{\alpha}\subseteq k[X_1,\dots,X_n]$ indexed by $\alpha$, we have $\mathcal{V}(\sum_{\alpha}I_{\alpha})=\mathcal{V}(\bigcup_{\alpha}I_{\alpha})=\bigcap_{\alpha}\mathcal{V}(I_{\alpha})$. This should be easy to verify, and it tells us that the arbitrary intersection of algebraic sets is algebraic. Next, we observe that for ideals $I,J\subseteq k[X_1,\dots,X_n]$, we have $\mathcal{V}(I\cap J)=\mathcal{V}(I\cdot J)=\mathcal{V}(I)\cup \mathcal{V}(J)$, where the set $I\cdot J=\{fg:f\in I,g\in J\}$. Let $p\in \mathcal{V}(I)\cup \mathcal{V}(J)$, assume without loss of generality that $p\in \mathcal{V}(I)$. For all $f\in I\cap J$, we have $f\in I$, so $f(p)=0$, hence $p\in \mathcal{V}(I\cap J)$, thus $\mathcal{V}(I)\cup\mathcal{V}(J)\subseteq \mathcal{V}(I\cap J)$. On the other hand, for each $fg\in I\cdot J$, we have $(fg)(p)=f(p)g(p)=0$, thus we have $\mathcal{V}(I)\cup\mathcal{V}(J)\subseteq \mathcal{V}(I\cdot J)$. Conversely, if $p\not\in \mathcal{V}(I)\cup \mathcal{V}(J)$, then there exists $f\in \mathcal{V}(I)$ and $g\in \mathcal{V}(J)$ such that $f(p)\ne 0$ and $g(p)\ne 0$, and hence $(fg)(p)=f(p)g(p)\ne 0$ as $k$ is a field. Since we know that $fg\in I\cdot J$ and $fg\in I\cap J$, we have $p\not\in \mathcal{V}(I\cdot J)$ and $p\not\in \mathcal{V}(I\cap J)$. Thus $\mathcal{V}(I\cap J),\mathcal{V}(I\cdot J)\subseteq \mathcal{V}(I)\cup\mathcal{V}(J)$.


\begin{definitionbox}An algebraic set $V$ is \textit{irreducible} if it cannot be written as $V=V_1\cup V_2$ where the algebraic sets $V_1,V_2\subset V$ properly, and such a set is called an \textit{(algebraic) varieties}.
\end{definitionbox}
\begin{lemma}An algebraic set $V$ is a variety iff $\mathcal{I}(V)$ is prime.
\begin{proof}Suppose that $\mathcal{I}(V)$ is not prime, that $fg\in \mathcal{I}(V)$ and $f\not\in \mathcal{I}(V)$.
\end{proof}
\end{lemma}
\end{document}
