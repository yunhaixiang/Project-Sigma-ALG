\RequirePackage{silence}
\WarningFilter{remreset}{The remreset package}
\documentclass[11pt]{book}
\usepackage{amsmath,amsthm,amssymb}
\usepackage[mathscr]{eucal}
\usepackage[hidelinks]{hyperref}
\usepackage[margin=1in]{geometry}
\usepackage{standalone}
\usepackage{enumitem}
\usepackage{fancyhdr}
\usepackage{setspace}
\usepackage{tikz}
\usepackage{mathpazo}
\usepackage{thmtools}
\usepackage{tikz}
\usepackage{tikz-cd}
\usepackage{xcolor}
\hypersetup{
    colorlinks,
    linkcolor={red!50!black},
    citecolor={blue!50!black},
    urlcolor={blue!80!black}
}
%\usepackage{tcolorbox}

%%%%%%%%%%%%% HEADERS %%%%%%%%%%%%%%%
\pagestyle{fancy}
\fancyhf{}
\rhead{\thepage}
\lhead{\leftmark}
\headheight=13.6pt

%%%%%%%%%%%%% ENVIRONMENTS %%%%%%%%%%%%%%%
\declaretheoremstyle[headfont=\color{blue!45!black!}\normalfont\bfseries]{bluestyle}
\declaretheoremstyle[headfont=\color{green!45!black!}\normalfont\bfseries]{greenstyle}
\declaretheoremstyle[headfont=\color{red!45!black!}\normalfont\bfseries]{redstyle}
\declaretheorem[numberwithin=section, style=definition]{definition}
\declaretheorem[sharenumber=definition, style=definition]{theorem}
\declaretheorem[sharenumber=definition, style=definition]{exercise}
\declaretheorem[sharenumber=definition, style=definition]{lemma}
\declaretheorem[sharenumber=definition, style=definition]{proposition}
\declaretheorem[sharenumber=definition, style=definition]{corollary}
\declaretheorem[sharenumber=definition, style=definition]{remark}
\declaretheorem[sharenumber=definition, style=definition]{axiom}
\declaretheorem[sharenumber=definition, style=definition]{example}
\declaretheorem[sharenumber=definition, style=definition]{result}
\declaretheorem[sharenumber=definition, style=definition]{problem}
\declaretheorem[sharenumber=definition, style=definition]{conjecture}
\declaretheorem[sharenumber=definition, style=definition]{algorithm}
\declaretheorem[sharenumber=definition, style=definition]{heuristic}
\declaretheorem[sharenumber=definition, style=definition]{motivation}
\declaretheorem[sharenumber=definition, style=definition]{intuition}
\declaretheorem[sharenumber=definition, style=definition]{anecdote}
\declaretheorem[sharenumber=definition, style=definition]{abbriviation}
\declaretheorem[sharenumber=definition, style=definition]{convention}

\declaretheorem[name={Definition},sharenumber=definition, shaded={margin=0.025\linewidth, textwidth=0.95\linewidth, bgcolor={red!3!white!}}, style=redstyle]{definitionbox}

\declaretheorem[name={Theorem},sharenumber=definition, shaded={margin=0.025\linewidth, textwidth=0.95\linewidth, bgcolor={blue!3!white!}}, style=bluestyle]{theorembox}

\declaretheorem[name={Exercise},sharenumber=definition, shaded={margin=0.025\linewidth, textwidth=0.95\linewidth, bgcolor={green!3!white!}}, style=greenstyle]{exercisebox}
\newenvironment{solution}{\begin{proof}[Solution]}{\end{proof}}
\newenvironment{abc}{\begin{enumerate}[label=(\alph*)]}{\end{enumerate}}
\renewcommand{\sectionautorefname}{\S}
\renewcommand{\subsectionautorefname}{\S}
\renewcommand{\subsubsectionautorefname}{\S}


%%%%%%%%%%%%% SHORTCUTS %%%%%%%%%%%%%%%
\DeclareMathOperator{\N}{\mathbf{N}}
\DeclareMathOperator{\Z}{\mathbf{Z}}
\DeclareMathOperator{\R}{\mathbf{R}}
\DeclareMathOperator{\F}{\mathbf{F}}
\DeclareMathOperator{\Q}{\mathbf{Q}}
\DeclareMathOperator{\GL}{\mathrm{GL}}
\DeclareMathOperator{\SL}{\mathrm{SL}}
\DeclareMathOperator{\lcm}{\operatorname{lcm}}
\DeclareMathOperator{\ord}{\operatorname{ord}}
\DeclareMathOperator{\sgn}{\operatorname{sgn}}
\DeclareMathOperator{\Aut}{\operatorname{Aut}}
\DeclareMathOperator{\Inn}{\operatorname{Inn}}
\DeclareMathOperator{\End}{\operatorname{End}}
\DeclareMathOperator{\stab}{\operatorname{stab}}
\DeclareMathOperator{\orb}{\operatorname{orb}}
%\DeclareMathOperator{\ker}{\operatorname{ker}}
\DeclareMathOperator{\im}{\operatorname{im}}
\DeclareMathOperator{\IM}{\operatorname{Im}}
\DeclareMathOperator{\RE}{\operatorname{Re}}
\DeclareMathOperator{\Span}{\operatorname{span}}
\DeclareMathOperator{\spec}{\operatorname{spec}}
\DeclareMathOperator{\Char}{\operatorname{char}}
\DeclareMathOperator{\Rank}{\operatorname{rank}}
\DeclareMathOperator{\proj}{\operatorname{proj}}
\DeclareMathOperator{\normal}{\trianglelefteq}


\title{
\vspace{-2.0cm}
\Large{Project Sigma}\\
\vspace{1cm}
\huge{\bf{Algebraic Geometry}}
\vspace{3cm}}
\author{Yunhai Xiang}
\date{\today}

\begin{document}


\maketitle
\doublespacing
\tableofcontents
\singlespacing
\newpage
%\setlength{\parskip}{10pt}
\chapter{Algebraic Varieties}
\section{Affine Algebraic Set}
Let $k$ be a field and $n\in\mathbf N$, then the \textit{affine space} $\mathbf A^n(k)$ of dimension $n$, or simply $\mathbf A^n$ if it does not cause confusion, is the same structure as the $n$-dimensional vector space $k^n$ over $k$, except with affine maps as morphisms, where an affine map is a linear map shifted by a constant. 

\begin{definition}
Suppose that $S\subseteq k[X_1,\dots,X_n]$ for some $n\in\mathbf N$, we define
\[\mathcal{V}(S)=\{x\in \mathbf A^n(k):\forall f\in S,\,f(x)=0\}\]
as the \textit{zero-locus} of $S$. A subset of $\mathbf A^n(k)$ that is the zero-locus of some $S$ is called \textit{(affine) algebraic}.
\end{definition}
For example, in $\mathbf{A}^2(\mathbf R)$, the sets $\mathcal{V}(\{Y\})$ and $\mathcal{V}(\{X\})$ are the $X$-axis and the $Y$-axis respectively, and the set $\mathcal{V}(\{X,Y\})$ is the origin. These are all examples of algebraic sets. As a non-example, the set $\{(\cos t,\sin t,t)\in\mathbf A^3(\mathbf R):t\in\mathbf R\}$ is not algebraic, as there is a line whose intersection with it is a infinite discrete set of points. Next, we claim that to find all algebraic sets, we need not consider all subsets of $k[X_1,\dots,X_n]$. Let $R$ be a commutative ring, we recall the following theorems.
\begin{theorem}$R$ is noetherian iff all ideals $I\subseteq R$ are finitely generated.
\end{theorem}
\begin{theorem}[Hilbert's Basis theorem]If $R$ is noetherian, then so is $R[X_1,\dots,X_n]$.
\end{theorem}

Therefore $k[X_1,\dots,X_n]$ is noetherian, and hence every for each $S\subseteq k[X_1,\dots,X_n]$, the ideal $\langle S\rangle$ is generated by some $f_1,\dots,f_m\in k[X_1,\dots,X_n]$. We claim that $\mathcal{V}(S)=\mathcal{V}(\langle S\rangle)=\mathcal{V}(\{f_1,\dots,f_m\})$. First, we know that $\mathcal{V}(\langle S\rangle)\subseteq \mathcal{V}(S)$ since $S\subseteq\langle S\rangle$. Conversely, suppose that $f\in\langle S\rangle$, then there exists $g_1,\dots,g_\ell\in S$ and $\lambda_1,\dots,\lambda_\ell\in k[X_1,\dots,X_n]$ with $f=\lambda_1g_1+\cdots+\lambda_{\ell}g_{\ell}$. Suppose that $p\in \mathcal{V}(S)$, then $f(p)=\lambda_1g_1(p)+\cdots+\lambda_{\ell}g_{\ell}(p)=0$. Since all $p\in \mathcal{V}(S)$ is a zero of all $f\in \langle S\rangle$, we must have $\mathcal{V}(S)\subseteq \mathcal{V}(\langle S\rangle)$, and hence $\mathcal{V}(S)=\mathcal{V}(\langle S\rangle)$. The equality $\mathcal{V}(\langle S\rangle)=\mathcal{V}(\{f_1,\dots,f_m\})$ is derived similarly. Therefore, it is a common abuse of notation to write $\mathcal{V}(f_1,\dots,f_n)$ instead of $\mathcal{V}(\{f_1,\dots,f_n\})$. Conversely, we can define an ideal $\mathcal{I}(X)$ for each $X\subseteq \mathbf A^n(k)$. 

\begin{definition}Let $X\subseteq \mathbf A^n(k)$, then define the ideal $\mathcal{I}(X)$ of $k[X_1,\dots,X_n]$ as 
\[\mathcal{I}(X)=\{f\in k[X_1,\dots,X_n]:\forall p\in X,\,f(p)=0\}\]
which is a well-defined ideal as we can verify easily.
\end{definition}
In fact, not only is $\mathcal{I}(X)$ an ideal, it is also a radical ideal. We recall that an radical ideal of a commutative ring $R$ is an ideal $I\subseteq R$ with $I=\sqrt{I}$ where $\sqrt{I}=\{r\in R:\exists m>0,\, r^m\in I\}$. In other words, a radical ideal is an ideal $I\subseteq R$ where for all $r\in R$, if $r^m\in I$ for some $m>0$, then $r\in I$. To see that $\mathcal{I}(X)$ is a radical ideal, note that if $f^m\in \mathcal{I}(X)$ for some $m>0$, then $f^m(p)=0$ for all $p\in X$, then $f(p)=0$ for all $p\in X$ as $k$ is a field, hence $f\in \mathcal{I}(X)$. 

Similar to how $\mathcal{V}(S)\subseteq \mathcal{V}(T)$ when $T\subseteq S\subseteq k[X_1,\dots,X_n]$, we easily have $\mathcal{I}(X)\subseteq \mathcal{I}(Y)$ when $Y\subseteq X\subseteq \mathbf A^n(k)$. Let $f\in S\subseteq k[X_1,\dots,X_n]$, then by definition $f$ vanishes on all of $\mathcal{V}(S)$, therefore $f\in \mathcal{I}(\mathcal{V}(S))$. Let $p\in X\subseteq \mathbf A^n(k)$, then by definition $p$ is a zero of all polynomials of $\mathcal{I}(X)$, therefore $p\in \mathcal{V}(\mathcal{I}(X))$. Hence we have $S\subseteq \mathcal{I}(\mathcal{V}(S))$ and $X\subseteq \mathcal{V}(\mathcal{I}(X))$. From these facts, we derive that $\mathcal{I}(X)= \mathcal{I}(\mathcal{V}(\mathcal{I}(X)))$ and $\mathcal{V}(S)= \mathcal{V}(\mathcal{I}(\mathcal V(S)))$. In fact, we have the following.
\begin{theorembox}[Hilbert's Nullstellensatz]There is a bijective correspondance
\[\{\textrm{radical\ ideals\ of\ }k[X_1,\dots,X_n]\}\longleftrightarrow\{\textrm{algebraic\ sets\ of\ }\mathbf A^n(k)\}\]
given by $I\mapsto \mathcal{V}(I)$ and $X\mapsto \mathcal{I}(X)$.
\end{theorembox}
whose proof we will delay until later in this note. This observation is central to algebraic geometry. 

We observe that for a nonempty family of ideals $I_{\alpha}\subseteq k[X_1,\dots,X_n]$ indexed by $\alpha$, we have $\mathcal{V}(\sum_{\alpha}I_{\alpha})=\mathcal{V}(\bigcup_{\alpha}I_{\alpha})=\bigcap_{\alpha}\mathcal{V}(I_{\alpha})$. This should be easy to verify, and it tells us that the arbitrary intersection of algebraic sets is algebraic. Next, we observe that for ideals $I,J\subseteq k[X_1,\dots,X_n]$, we have $\mathcal{V}(I\cap J)=\mathcal{V}(I\cdot J)=\mathcal{V}(I)\cup \mathcal{V}(J)$, where the set $I\cdot J=\{fg:f\in I,g\in J\}$. Suppose that $p\in \mathcal{V}(I)\cup \mathcal{V}(J)$, assume without loss of generality that $p\in \mathcal{V}(I)$. For all $f\in I\cap J$, we have $f\in I$, so $f(p)=0$, hence $p\in \mathcal{V}(I\cap J)$, thus $\mathcal{V}(I)\cup\mathcal{V}(J)\subseteq \mathcal{V}(I\cap J)$. On the other hand, for each $fg\in I\cdot J$, we have $(fg)(p)=f(p)g(p)=0$, thus we have $\mathcal{V}(I)\cup\mathcal{V}(J)\subseteq \mathcal{V}(I\cdot J)$. Conversely, if $p\not\in \mathcal{V}(I)\cup \mathcal{V}(J)$, then there exists $f\in \mathcal{V}(I)$ and $g\in \mathcal{V}(J)$ such that $f(p)\ne 0$ and $g(p)\ne 0$, and hence $(fg)(p)=f(p)g(p)\ne 0$ as $k$ is a field. Since we know that $fg\in I\cdot J$ and $fg\in I\cap J$, we have $p\not\in \mathcal{V}(I\cdot J)$ and $p\not\in \mathcal{V}(I\cap J)$. Thus $\mathcal{V}(I\cap J),\mathcal{V}(I\cdot J)\subseteq \mathcal{V}(I)\cup\mathcal{V}(J)$, and hence we completed the proof. Moreover, since $IJ\subseteq I\cap J$, we have $\mathcal{V}(I)\cup\mathcal{V}(J)=\mathcal{V}(I\cap J)\subseteq\mathcal{V}(IJ)$, and since $I\cdot J\subseteq IJ$, we have $\mathcal{V}(IJ)\subseteq\mathcal{V}(I\cdot J)=\mathcal{V}(I)\cup\mathcal{V}(J)$. Hence $\mathcal{V}(IJ)=\mathcal{V}(I)\cup\mathcal{V}(J)$ as well.


\begin{definitionbox}An algebraic set $V$ is \textit{irreducible} if it cannot be written as $V=V_1\cup V_2$ where the algebraic sets $V_1,V_2\subset V$ properly, and such a set is called an \textit{(algebraic) variety}.
\end{definitionbox}
\begin{exercise}If $\mathcal{I}(X)=\mathcal{I}(Y)$ for algebraic sets $X,Y$, then $X=Y$.
\end{exercise}
\begin{lemma}An algebraic set $V$ is a variety iff $\mathcal{I}(V)$ is prime.
\begin{proof}Suppose that $\mathcal{I}(V)$ is not prime, that $fg\in \mathcal{I}(V)$ and $f,g\not\in \mathcal{I}(V)$. We claim that 
\[V=(V\cap \mathcal{V}(f))\cup (V\cap \mathcal{V}(g))\]
Let $p\in V$, then $(fg)(p)=f(p)g(p)=0$, thus $f(p)=0$ or $g(p)=0$ since $k$ is a field. Hence we have $p\in \mathcal{V}(f)$ or $p\in \mathcal{V}(g)$. Therefore $V\subseteq (V\cap \mathcal{V}(f))\cup (V\cap \mathcal{V}(g))$, the other direction $(V\cap \mathcal{V}(f))\cup (V\cap \mathcal{V}(g))\subseteq V$ is obvious. Since $f\not\in \mathcal{I}(V)$, exists $p\in V$ with $f(p)\ne 0$. Thus $p\not\in \mathcal{V}(f)$. Thus $V\ne V\cap \mathcal{V}(f)$. Similarly, $V\ne V\cap \mathcal{V}(g)$, so $V$ is reducible. Conversely, assume $V=V_1\cup V_2$ where $V_1,V_2\subset V$ properly. We have $\mathcal{I}(V)\subset \mathcal{I}(V_1),\mathcal{I}(V_2)$ properly. Choose $f\in \mathcal{I}(V_1)\setminus \mathcal{I}(V)$ and $g\in \mathcal{I}(V_2)\setminus \mathcal{I}(V)$. For $p\in V$, we have $p\in V_1$ or $p\in V_2$, thus $f(p)=0$ or $g(p)=0$, so $(fg)(p)=f(p)g(p)=0$. Hence $fg\in \mathcal{I}(V)$, so $\mathcal{I}(V)$ is not prime.
\end{proof}
\end{lemma}


Take, for example, the algebraic set $V=\mathcal{V}(f,g)\subseteq \mathbf A^3(\mathbf R)$ where $f(x,y,z)=x^2+y^2+z^2-4$ and $g(x,y,z)=y^2+z^2-1$. Then $V$ is the intersection of the sphere of radius $2$, and the cylinder of radius $1$. In fact, we have a decomposition of $V$
\[V=\mathcal{V}(x-\sqrt{3},y^2+z^2-1)\cup \mathcal{V}(x+\sqrt{3},y^2+z^2-1)\]
into algebraic varieties. This is easy to visualize and check that it is true. In fact, we can do even better. We will show that each algebaic set has a unique decomposition into algebraic varieties. Suppose that $R$ is a commutative ring, we recall the following theorem.
\begin{theorem}\label{thm:min}$R$ is noetherian iff every nonempty set of ideals has a maximal element.
\end{theorem}
\begin{theorem}If $V$ is an algebraic set, then $V$ has a unique decomposition $V=V_1\cup\cdots\cup V_m$, where $V_1,\dots,V_m$ are varieties such that no one of them is contained in another.
\begin{proof}
Suppose that $\mathcal{L}$ is the set of algebraic sets that do not admit a finite variety decomposition, we will show that $\mathcal{L}=\emptyset$. Suppose the contrary, then $\mathcal{L}$ has a minimal element $V$ w.r.t inclusion by \autoref{thm:min} on $\mathcal{I}[\mathcal{L}]$. Since $V\in A$, we have $V$ is reducible, hence $V=V_1\cup V_2$ with algebraics sets $V_1,V_2\subset V$ properly. Since $V$ is minimal, we must have $V_1,V_2\not\in \mathcal{L}$. Thus $V_1,V_2$ admit finite variety decompositions, contradiction. Next, we show the uniqueness. Let $V=V_1\cup\cdots\cup V_m=W_1\cup\dots\cup W_h$ be decompositions, then $V_i=(V_i\cap W_1)\cup\cdots\cup (V_i\cap W_h)$, which by the irreducibility of $V_i$, tells us that $V_i\subseteq W_{\sigma(i)}$ for some $\sigma(i)$. Similarly $W_j\subseteq V_{\delta(j)}$ for some $\delta(j)$. Thus $V_i\subseteq W_{\sigma(i)}\subseteq V_{\delta(\sigma(i))}$. However, $V_i\subseteq V_{\delta(\sigma(i))}$ implies that $V_i=V_{\delta(\sigma(i))}$, so $i=\delta(\sigma(i))$ and $V_i=W_{\sigma(i)}$. 
\end{proof}
\end{theorem}
By developing this general theory further, we will take a look at the affine plane $\mathbf A^2(k)$ and find all its algebraic subsets. From what we showed above, it suffice to find all algebraic varieties. From there, we will conclude that the irreducible algebraic subsets of $\mathbf A^2(k)$ are the empty set $\emptyset$, the whole space $\mathbf A^2(k)$, single points, and irreducible plane curves $\mathcal{V}(F)$, where $F$ is an irreducible polynomial such that $\mathcal{V}(F)$ is infinite.
\begin{proposition}Let $F,G\in k[X,Y]$ be coprime, then $\mathcal{V}((F,G))$ is a finite set of points.
\begin{proof}
Since $F,G$ are coprime in $k[X,Y]$, they are coprime in $k(X)[Y]$ as well. Since $k(X)[Y]$ is PID, $(F,G)=(1)$, hence $RF+SG=1$ for some $R,S\in k(X)[Y]$. Choose nonzero $D\in k[X]$ with $DR=A$ and $DS=A$ such that $A,B\in k[X,Y]$, then $AF+BG=D$. If $(a,b)\in \mathcal{V}((F,G))$ then $D(a)=0$, but $D$ has only finite number of zeros. 
\end{proof}
\end{proposition}
We claim that if $F\in k[X,Y]$ is irreducible and $\mathcal{V}((F))$ is infinite, then $\mathcal{I}(\mathcal{V}((F)))=(F)$ and $\mathcal{V}(F)$ is irreducible. If $G\in \mathcal{I}(\mathcal{V}((F)))$ then $\mathcal{V}((F,G))$ is infinite, so $G\in (F)$. Hence $(F)\subseteq \mathcal{I}(\mathcal{V}((F)))$. 
\begin{proposition}Suppose that $k$ is algebraically closed and $F$ is a nonconstant polynomial in $k[X,Y]$ with decomposition $F=F_1^{n_1}\cdots F_r^{n_r}$, then $\mathcal{V}((F))=\mathcal{V}((F_1))\cup\cdots\cup \mathcal{V}((F_r))$ is the decomposition of $\mathcal{V}((F))$, and $\mathcal{I}(\mathcal{V}((F)))=(F_1\cdots F_r)$.
\end{proposition}
\chapter{Sheaves and Schemes}
\section{Presheaves and Sheaves}
\begin{definition}A \textit{presheaf} $\mathscr{F}$ of sets on a topological space $X$ contains the following information:
\begin{enumerate}[label=(\roman*)]
	\item for each open $U\subseteq X$, a set $\mathscr{F}(U)$, the elements of which are called the \textit{sections} of $\mathscr{F}$ over $U$,
	\item for each inclusion $U\hookrightarrow V$ of open sets, a \textit{restriction} map $\mathrm{Res}^V_U:\mathscr{F}(V)\rightarrow\mathscr{F}(U)$, such that
	\begin{itemize}
		\item for all $U\subseteq X$, the map $\mathrm{Res}^U_U$ is the identity map,
		\item if $U\hookrightarrow V \hookrightarrow W$ are inclusions of open sets then the diagram
		\[
		\begin{tikzcd}[sep=huge]
			\mathscr{F}(W) \arrow[r,"\mathrm{Res}^W_V"] \arrow[rr, "\mathrm{Res}^W_U", bend left, shift left=1] & \mathscr{F}(V) \arrow[r,"\mathrm{Res}^V_U"] & \mathscr{F}(U)
		\end{tikzcd}
		\]
		commutes, in other words, $\mathrm{Res}^W_U=\mathrm{Res}^V_U\circ \mathrm{Res}^W_V$
	\end{itemize}
\end{enumerate}
Moreover, suppose that $\mathscr{F},\mathscr{G}$ are presheaves of sets on $X$, then a morphism $\phi:\mathscr{F}\rightarrow\mathscr{G}$ is the data of a map $\phi(U):\mathscr{F}(U)\rightarrow\mathscr{G}(U)$ for each open $U\subseteq X$ such that the diagram
\[
\begin{tikzcd}[sep = huge]
\mathscr{F}(V) \arrow[r, "\phi(V)"] \arrow[d, "\mathrm{Res}^V_U"'] & \mathscr{G}(V) \arrow[d, "\mathrm{Res}^V_U"] \\
\mathscr{F}(U) \arrow[r, "\phi(U)"]                 & \mathscr{G}(U)               
\end{tikzcd}\]
commutes whenever $U\hookrightarrow V$ is an inclusion of open sets (where we abuse the notation $\mathrm{Res}^V_U$).
\end{definition}
In fact, we can see that a presheaf of sets on a topological space is exactly the same information of a contravariant functor from the category of open sets of $X$, which is the category consisting of all open sets of $X$ as objects and inclusions as morphisms, to the category of sets. 
\begin{definitionbox}
Let $\mathcal{C},\mathcal{D}$ be categories where $\mathcal{C}$ is small, then a $\mathcal{D}$-valued presheaf on $\mathcal{C}$ is a functor $\mathcal{C}^{\mathrm{op}}\rightarrow\mathcal{D}$, and the category of $\mathcal{D}$-valued presheaf on $\mathcal{C}$ is the functor category $[\mathcal{C}^{\mathrm{op}},\mathcal{D}]$.
\end{definitionbox}

Many examples of presheaves come from different classes of functions (with the obvious restriction map). For example, we can define $\mathscr{F}(U)=\mathscr{C}^0(U)$ as the set of continuous real-valued functions on $U$, or we can define $\mathscr{F}(U)=\mathscr{B}(U)$ as the set of complex-valued functions on $U$ that are bounded. There is also the \textit{constant presheaf} associated with a set $S$ where $\mathscr{F}(U)=S$ for all open $U\subseteq X$, and the retriction map is the identity map on $S$.
\begin{definition}Let $\mathscr{F}$ be a presheaf on a topological space $X$, then we define the \textit{stalk} $\mathscr{F}_p$ of $\mathscr{F}$ at a point $p\in X$, whose elements are called \textit{germs} of $\mathscr{F}$ at $p$, as
\[\mathscr{F}_p=\left.\left\{(f,U):\begin{array}{c}f\in\mathscr{F}(U),\textrm{\ and\ }U\subseteq X\textrm{\ is}\\\textrm{an\ open\ neighborhood\ of\ }p\end{array}\right\} \middle/ \left\{\begin{array}{c}(f,U)\sim (g,V)\textrm{\ if\ and\ only\ if\ exists\ }\\W\subseteq U\cap V\textrm{\ such\ that\ }\mathrm{Res}^U_Wf=\mathrm{Res}^V_Wg\end{array}\right\}\right.\]
In other words, $\mathscr{F}_p$ is the colimit $\displaystyle\lim_{\longrightarrow}\mathscr{F}(U)$ indexed over open neighborhoods $U$ of $p$ with inclusion.
\end{definition}
Moreover, in many cases like the examples mentioned above where $\mathscr{F}(U)$ is a set of functions, we can define a ring structure on the stalk $\mathscr{F}_p$ by defining $[(f,U)]+[(g,V)]=[(f+g,U\cap V)]$ and $[(f,U)][(g,V)]=[(fg,U\cap V)]$. We can check that this is well-defined. Hence, in many situations, we will consider $\mathscr{F}_p$ as a ring (which is an abuse of notation). Stalks capture the local properties of a (pre)sheaf, which we will elaborate later.
\begin{definition}A presheaf $\mathscr{F}$ on a topological space $X$ is a \textit{sheaf} if for any open set $U\subseteq X$ and any open cover $\{U_i\}_{i\in\mathcal{I}}$ of $U$ the following two axioms are satisfied:
\begin{enumerate}[label=(\roman*)]
	\item \textit{identity axiom}: if $f,g\in\mathscr{F}(U)$, then $\mathrm{Res}^U_{U_i}f=\mathrm{Res}^U_{U_i}g$ for all $i\in\mathcal{I}$ implies that $f=g$, and
	\item \textit{gluability axiom}: if $f_i\in \mathscr{F}(U_i)$ for all $i\in\mathcal{I}$ is such that $\mathrm{Res}^{U_i}_{U_i\cap U_j}f_i=\mathrm{Res}^{U_j}_{U_i\cap U_j}f_j$ for all $i,j\in\mathcal{I}$, then there exists some $f\in\mathscr{F}(U)$ such that $\mathrm{Res}^U_{U_i}f=f_i$ for all $i\in\mathcal{I}$.
\end{enumerate}
\end{definition}

\end{document}
