\RequirePackage{silence}
\WarningFilter{remreset}{The remreset package}
\documentclass[11pt]{book}
\usepackage{amsmath,amsthm,amssymb}
\usepackage[mathscr]{eucal}
\usepackage[hidelinks]{hyperref}
\usepackage[margin=1in]{geometry}
\usepackage{standalone}
\usepackage{enumitem}
\usepackage{fancyhdr}
\usepackage{setspace}
\usepackage{tikz}
\usepackage{mathpazo}
\usepackage{thmtools}
\usepackage{tikz}
\usepackage{tikz-cd}
\usepackage{xcolor}
\hypersetup{
    colorlinks,
    linkcolor={red!50!black},
    citecolor={blue!50!black},
    urlcolor={blue!80!black}
}
%\usepackage{tcolorbox}

%%%%%%%%%%%%% HEADERS %%%%%%%%%%%%%%%
\pagestyle{fancy}
\fancyhf{}
\rhead{\thepage}
\lhead{\leftmark}
\headheight=13.6pt

%%%%%%%%%%%%% ENVIRONMENTS %%%%%%%%%%%%%%%
\declaretheoremstyle[headfont=\color{blue!45!black!}\normalfont\bfseries]{bluestyle}
\declaretheoremstyle[headfont=\color{green!45!black!}\normalfont\bfseries]{greenstyle}
\declaretheoremstyle[headfont=\color{red!45!black!}\normalfont\bfseries]{redstyle}
\declaretheorem[numberwithin=section, style=definition]{definition}
\declaretheorem[sharenumber=definition, style=definition]{theorem}
\declaretheorem[sharenumber=definition, style=definition]{exercise}
\declaretheorem[sharenumber=definition, style=definition]{lemma}
\declaretheorem[sharenumber=definition, style=definition]{proposition}
\declaretheorem[sharenumber=definition, style=definition]{corollary}
\declaretheorem[sharenumber=definition, style=definition]{remark}
\declaretheorem[sharenumber=definition, style=definition]{axiom}
\declaretheorem[sharenumber=definition, style=definition]{example}
\declaretheorem[sharenumber=definition, style=definition]{result}
\declaretheorem[sharenumber=definition, style=definition]{problem}
\declaretheorem[sharenumber=definition, style=definition]{conjecture}
\declaretheorem[sharenumber=definition, style=definition]{algorithm}
\declaretheorem[sharenumber=definition, style=definition]{heuristic}
\declaretheorem[sharenumber=definition, style=definition]{motivation}
\declaretheorem[sharenumber=definition, style=definition]{intuition}
\declaretheorem[sharenumber=definition, style=definition]{anecdote}
\declaretheorem[sharenumber=definition, style=definition]{abbriviation}
\declaretheorem[sharenumber=definition, style=definition]{convention}

\declaretheorem[name={Definition},sharenumber=definition, shaded={margin=0.025\linewidth, textwidth=0.95\linewidth, bgcolor={red!3!white!}}, style=redstyle]{definitionbox}

\declaretheorem[name={Theorem},sharenumber=definition, shaded={margin=0.025\linewidth, textwidth=0.95\linewidth, bgcolor={blue!3!white!}}, style=bluestyle]{theorembox}

\declaretheorem[name={Exercise},sharenumber=definition, shaded={margin=0.025\linewidth, textwidth=0.95\linewidth, bgcolor={green!3!white!}}, style=greenstyle]{exercisebox}
\newenvironment{solution}{\begin{proof}[Solution]}{\end{proof}}
\newenvironment{abc}{\begin{enumerate}[label=(\alph*)]}{\end{enumerate}}
\renewcommand{\sectionautorefname}{\S}
\renewcommand{\subsectionautorefname}{\S}
\renewcommand{\subsubsectionautorefname}{\S}
\renewcommand{\thetable}{\thechapter.\Alph{table}}

%%%%%%%%%%%%% SHORTCUTS %%%%%%%%%%%%%%%
\DeclareMathOperator{\N}{\mathbf{N}}
\DeclareMathOperator{\Z}{\mathbf{Z}}
\DeclareMathOperator{\R}{\mathbf{R}}
\DeclareMathOperator{\F}{\mathbf{F}}
\DeclareMathOperator{\Q}{\mathbf{Q}}
\DeclareMathOperator{\GL}{\mathrm{GL}}
\DeclareMathOperator{\SL}{\mathrm{SL}}
\DeclareMathOperator{\lcm}{\operatorname{lcm}}
\DeclareMathOperator{\ord}{\operatorname{ord}}
\DeclareMathOperator{\sgn}{\operatorname{sgn}}
\DeclareMathOperator{\Aut}{\operatorname{Aut}}
\DeclareMathOperator{\Inn}{\operatorname{Inn}}
\DeclareMathOperator{\End}{\operatorname{End}}
\DeclareMathOperator{\stab}{\operatorname{stab}}
\DeclareMathOperator{\orb}{\operatorname{orb}}
%\DeclareMathOperator{\ker}{\operatorname{ker}}
\DeclareMathOperator{\im}{\operatorname{im}}
\DeclareMathOperator{\IM}{\operatorname{Im}}
\DeclareMathOperator{\RE}{\operatorname{Re}}
\DeclareMathOperator{\Span}{\operatorname{span}}
\DeclareMathOperator{\spec}{\operatorname{spec}}
\DeclareMathOperator{\Char}{\operatorname{char}}
\DeclareMathOperator{\Rank}{\operatorname{rank}}
\DeclareMathOperator{\proj}{\operatorname{proj}}
\DeclareMathOperator{\normal}{\trianglelefteq}


\title{
\vspace{-2.0cm}
\Large{Project Sigma}\\
\vspace{1cm}
\huge{\bf{Algebraic Geometry}}
\vspace{3cm}}
\author{Yunhai Xiang}
\date{\today}

\begin{document}


\maketitle
\doublespacing
\tableofcontents
\singlespacing
\newpage
%\setlength{\parskip}{10pt}
\chapter{Preliminaries}

\section{Categories and Functors}
To discuss algebraic geometry cleanly and efficiently, we need to introduce category theory. Given sets $X$ and $Y$, from elementary set theory, a \textit{function} $f:X\rightarrow Y$ consists of the information of its \textit{domain} $X$, \textit{codomain} $Y$, and \textit{graph} $G\subseteq X\times Y$ where for each $x\in X$ there exists unique $y\in Y$ such that $(x,y)\in G$. In category theory, we generalize the notion of a function. Suppose we are given two \textit{objects} $X$ and $Y$, a \textit{morphism} $f:X\rightarrow Y$ consists of the information of its \textit{source} $X$, \textit{target} $Y$, and possibly some additional information. It is meaningless to talk about a morphism without talking about the category it belongs to, therefore we will make these terminologies precise.
\begin{definition}A \textit{category} $\mathcal C$ is the following data
\begin{enumerate}[label=(\roman*)]
	\item A class, also denoted $\mathcal C$, whose elements are called \textit{objects} of $\mathcal C$,
	\item for each pair of objects $X,Y\in \mathcal C$, a class $\mathrm{Mor}(X,Y)$ of \textit{morphisms} from $X$ to $Y$, such that each $f\in \mathrm{Mor}(X,Y)$ consists of the data of its \textit{source} $X$, \textit{target} $Y$, and possibly additional data,
	\item for each triple of object $X,Y,Z\in\mathcal C$, a \textit{composition} function 
	\[(-\circ -):\mathrm{Mor}(Y,Z)\times\mathrm{Mor}(X,Y)\rightarrow \mathrm{Mor}(X,Z)\]
	such that the following axioms are satisfied
	\begin{itemize}
		\item \textit{identity axiom}: for each $X\in\mathcal{C}$, exists $\mathbf 1_X\in\mathrm{Mor}(X,X)$ s.t. if $Y\in\mathcal C$, for all $f\in\mathrm{Mor}(X,Y)$ we have $f\circ \mathbf 1_X=f$ and for all $g\in \mathrm{Mor}(Y,X)$ we have $\mathbf 1_X\circ g=g$.
		\item \textit{associativity axiom}: for each quadruple of objects $X,Y,Z,W\in \mathcal C$, given $f\in \mathrm{Mor}(X,Y)$, $g\in \mathrm{Mor}(Y,Z)$, and $h\in\mathrm{Mor}(Z,W)$, we have $(h\circ g)\circ f=h\circ (g\circ f)$.
	\end{itemize}
\end{enumerate}
Moreover, we say that the category is \textit{locally small} if $\mathrm{Mor}(X,Y)$ is a set for all objects $X,Y$, and \textit{small} if it is locally small and the class of objects is also a set. By convention, we use $f:X\rightarrow Y$ to denote $f\in\mathrm{Mor}(X,Y)$ when there is no risk of confusion. 
\end{definition}
\begin{example}The quintessential example to keep in mind is the category $\mathbf{Set}$ whose objects are sets and the morphisms are just functions between sets with the usual composition. In fact, many categories we encounter are such that their objects are ``sets with extra structures'' and morphisms are ``functions with structure-preserving properties''. They are known as \textit{concrete categories} but we will not make this term precise or use this terminology. In the next page, we will provide a list of the definitions of these type of categories.
\end{example}

\begin{table}[ht]
\centering
\caption{Table of some useful categories}
\vspace{3mm}
\begin{tabular}[t]{l|c|c|c}
\textit{name of category} & \textit{notation} & \textit{objects} & \textit{morphisms}\\
\hline
category of sets &$\mathbf{Set}$ & sets & functions\\
category of groups & $\mathbf{Grp}$ & groups & group homomorphisms\\
category of abelian groups & $\mathbf{Ab}$ & abelian groups & group homomorphisms \\
category of rings & $\mathbf{CRing}$ & (commutative) rings & ring homomorphisms\\
category of posets & $\mathbf{Pos}$ & posets & increasing functions\\
category of graphs & $\mathbf{Graph}$ & (simple) graphs & graph homomorphisms\\
category of topological spaces & $\mathbf{Top}$ & topological spaces & continuous functions\\
category of Hausdorff spaces & $\mathbf{Haus}$ & Hausdorff spaces & continuous functions\\
category of $\mathscr{C}^\ell$ manifolds & $\mathbf{Man}_{\ell}$ & $\mathscr{C}^\ell$ manifolds & $\mathscr{C}^\ell$ maps\\
category of affine spaces over $k$ & $\mathbf{Aff}_k$ & affine spaces over $k$ & affine maps\\
category of vector spaces over $k$ & $\mathbf{Vect}_k$ & vector spaces over $k$ & linear maps\\
category of modules over $R$ & $\mathbf{Mod}_R$ & modules over $R$ & module homomorphisms\\
\end{tabular}
\label{tab:cat}
\end{table}
\begin{remark}Throughout this note, the term rings refers to commutative rings (with unit), thus we will not introduce the category $\mathbf{Ring}$ which includes possibly non-commutative rings.
\end{remark}
\begin{definition}Suppose $\mathcal C$ is a category with $X,Y\in\mathcal C$, a morphism $f:X\rightarrow Y$ is a(n)
\begin{enumerate}[label=(\roman*)]
	\item \textit{endomorphism} if $X=Y$,
	\item \textit{monomorphism} if left-cancellable, i.e. $f\circ g=f\circ h$ implies $g=h$ for $g,h:Z\rightarrow X$ and $Z\in\mathcal C$,
	\item \textit{epimorphism} if right-cancellable, i.e. $g\circ f=h\circ f$ implies $g=h$ for $g,h:X\rightarrow Z$ and $Z\in\mathcal C$,
	\item \textit{split monomorphism} if it admits left-inverse, i.e. there exists $g:Y\rightarrow X$ such that $g\circ f=\mathbf 1_X$,
	\item \textit{split epimorphism} if it admits right-inverse, i.e there exists $g:Y\rightarrow X$ such that $f\circ g=\mathbf 1_Y$,
	\item \textit{bimorphism} if it's both a monomorphism and an epimorphism,
	\item \textit{isomorphism} if it's both a split monomorphism and a split epimorphism,
	\item \textit{automorphism} if it's both an isomorphism and an endomorphism. 
\end{enumerate}
Moreover, a category where all morphisms are isomorphisms is a \textit{groupoid}. 
\end{definition}
In $\mathbf{Set}$, the monomorphisms are exactly the injections and epimorphisms exactly surjections. In ``concrete categories'', the injective morphisms are monic and the surjective morphisms are epic, but the converes need not hold. A classical counterexample is the canonical inclusion $\mathbf Z\hookrightarrow\mathbf Q$ in $\mathbf{CRing}$, which is epic but is clearly not surjective. This is also an example of an epimorphism that is not split. Straightforwardly, split monomorphisms are monomorhpisms and split epimorphisms are epimorphisms. The converses need not hold as we have seen in the previous example. Hence, a bimorphism need not be an isomorphism, even though an isomorphism must be a bimorphism. 

Even though most categories I've shown so far are ``concrete categories'', there are categories where the morphisms are not functions. A classical example is the category $\mathbf{Rel}$ where the objects are sets and a morphism $f:X\rightarrow Y$ is just a subset $f\subseteq X\times Y$. Given two morphisms $f:X\rightarrow Y$ and $g:Y\rightarrow Z$, define the composition $g\circ f:X\rightarrow Z$ as the set $g\circ f=\{(a,c)\in X\times Z:\exists b\in Y,(a,b)\in f\textrm{\ and\ }(b,c)\in g\}$. We can easily check that this category is well defined, and moreover, the identity on an object is the equality relation as one would expect.

\begin{definition}Let $\mathcal C$ and $\mathcal D$ be categories, a \textit{(covariant) functor} $\mathscr{F}:\mathcal C\rightarrow\mathcal{D}$ is the following data,
\begin{enumerate}[label=(\roman*)]
	\item for each object $X\in\mathcal C$, an object $\mathscr{F}(X)\in\mathcal D$,
	\item for each morphism $f:X\rightarrow Y$ of $\mathcal C$, a morphism $\mathscr{F}f:\mathscr{F}(X)\rightarrow\mathscr{F}(Y)$ of $\mathcal D$, s.t.
	\begin{itemize}
		\item $\mathscr{F}\mathbf{1}_X=\mathbf 1_{\mathscr{F}(X)}$ for all $X\in\mathcal C$,
		\item $\mathscr{F}(g\circ f)=\mathscr{F}g\circ\mathscr{F}f$ for all morphisms $f:X\rightarrow Y$ and $g:Y\rightarrow Z$.
	\end{itemize}
\end{enumerate}
A \textit{contravariant functor} $\mathscr{F}:\mathcal C\rightarrow\mathcal D$ is defined similarly, except for each morphism $f:X\rightarrow Y$ of $\mathcal C$ we assign a morphism $\mathscr{F}f:\mathscr{F}(Y)\rightarrow\mathscr{F}(X)$, reversing all the arrows. We can think of a contravariant functor $\mathscr{F}:\mathcal{C}\rightarrow\mathcal D$ as a covariant functor $\mathscr{F}:\mathcal{C}^{\mathrm{op}}\rightarrow\mathcal D$ where we define the \textit{opposite category} $\mathcal C^{\mathrm{op}}$ as the category with the same objects as $\mathcal{C}$ but $\mathrm{Mor}_{\mathcal{C}^{\mathrm{op}}}(X,Y)=\mathrm{Mor}_{\mathcal{C}}(Y,X)$ for all objects $X,Y\in\mathcal C^{\mathrm{op}}$. Moreover, we say that the functor $\mathscr{F}:\mathcal C\rightarrow \mathcal D$ is \textit{faithful} (resp. \textit{full}) if the function (where we abuse notation)
$\mathscr{F}:\mathrm{Mor}_{\mathcal C}(X,Y)\rightarrow\mathrm{Mor}_{\mathcal D}(\mathscr F(X),\mathscr F(Y))$ given by $f\mapsto \mathscr{F}f$ is injective (resp. surjective) for all objects $X,Y\in\mathcal C$. We say that $\mathscr{F}$ is \textit{fully faithful} if it's both full and faithful. The compositions of functors is the obvious element-wise composition
\end{definition}
The reason we use the notation $\mathscr{F}:\mathcal C\rightarrow\mathcal D$ is that we view functors as ``morphisms'' between categories, but there is just this small issue: there can not possibly be a category of all categories, as that would lead to set-theoretic troubles. We can, however, define the category $\mathbf{Cat}$ whose objects are all small categories with functors as morphisms. It should be noted that even if one of $\mathcal C$ or $\mathcal D$ is not small, we can still talk about functors between them and and use the notation $\mathscr{F}:\mathcal C\rightarrow\mathcal D$ even if technically $\mathscr{F}$ is not a morphism of any category we defined. 
\begin{example}Define $\mathbf{Top}_{\bullet}$ as the category of pointed topological spaces, i.e. the category where the objects are topological spaces with a choice of a base point and the morphisms are continuous functions that preserve the base points. Define functor $\pi_1:\mathbf{Top}_{\bullet}\rightarrow\mathbf{Grp}$ which assigns to each pointed topological space $\langle x,X\rangle$ the fundamental group $\pi_1(x,X)$. For each $f:\langle x,X\rangle\rightarrow \langle y,Y\rangle$, we assign $\pi_1f:\pi_1(x,X)\rightarrow\pi_1(y,Y)$ the induced homomorphism $[\varphi]\mapsto [f\circ\varphi]$.
\end{example}

In category theory, we like to draw diagrams a lot. To draw a diagram in a category $\mathcal C$, we pick out some objects in $\mathcal C$ and pick out some morphisms between these chosen objects, we write down symbols for the chosen objects and draw arrows representing the chosen morphisms, e.g.



The reason we require a functor to preserve identity and preserve composition is precisely because we want functors to ``preserve diagrams''.
\begin{definition}Let $\mathcal C,\mathcal D$ be categories, $\mathscr{F},\mathscr{G}:\mathcal{C}\rightarrow\mathcal D$ functors. A \textit{natural transformation} $\phi:\mathscr{F}\rightarrow\mathscr{G}$ is the data of a map $\phi(X):\mathscr{F}(X)\rightarrow\mathscr{G}(X)$ for each $X\in\mathcal C$ s.t. the diagram

commutes, i.e. $\phi(Y)\circ \mathscr{G}f=\mathscr{F}f\circ\phi(X)$ for all morphisms $f:X\rightarrow Y$ and objects $X,Y\in\mathcal C$.
\end{definition}



\section{Limits}
\section{Additive and Abelian Categories}
\section{Varieties}
Let $k$ be a field and $n\in\mathbf N$, then the \textit{affine space} $\mathbf A^n(k)$ of dimension $n$, or simply $\mathbf A^n$ if it does not cause confusion, is the same structure as the $n$-dimensional vector space $k^n$ over $k$, except with affine maps as morphisms, where an affine map is a linear map shifted by a constant. 

\begin{definition}
Suppose that $S\subseteq k[X_1,\dots,X_n]$ for some $n\in\mathbf N$, we define
\[\mathcal{V}(S)=\{x\in \mathbf A^n(k):\forall f\in S,\,f(x)=0\}\]
as the \textit{zero-locus} of $S$. A subset of $\mathbf A^n(k)$ that is the zero-locus of some $S$ is called \textit{(affine) algebraic}.
\end{definition}
For example, in $\mathbf{A}^2(\mathbf R)$, the sets $\mathcal{V}(\{Y\})$ and $\mathcal{V}(\{X\})$ are the $X$-axis and the $Y$-axis respectively, and the set $\mathcal{V}(\{X,Y\})$ is the origin. These are all examples of algebraic sets. As a non-example, the set $\{(\cos t,\sin t,t)\in\mathbf A^3(\mathbf R):t\in\mathbf R\}$ is not algebraic, as there is a line whose intersection with it is a infinite discrete set of points. Next, we claim that to find all algebraic sets, we need not consider all subsets of $k[X_1,\dots,X_n]$. Let $R$ be a commutative ring, we recall the following theorems.
\begin{theorem}$R$ is noetherian iff all ideals $I\subseteq R$ are finitely generated.
\end{theorem}
\begin{theorem}[Hilbert's Basis theorem]If $R$ is noetherian, then so is $R[X_1,\dots,X_n]$.
\end{theorem}

Therefore $k[X_1,\dots,X_n]$ is noetherian, and hence every for each $S\subseteq k[X_1,\dots,X_n]$, the ideal $\langle S\rangle$ is generated by some $f_1,\dots,f_m\in k[X_1,\dots,X_n]$. We claim that $\mathcal{V}(S)=\mathcal{V}(\langle S\rangle)=\mathcal{V}(\{f_1,\dots,f_m\})$. First, we know that $\mathcal{V}(\langle S\rangle)\subseteq \mathcal{V}(S)$ since $S\subseteq\langle S\rangle$. Conversely, suppose that $f\in\langle S\rangle$, then there exists $g_1,\dots,g_\ell\in S$ and $\lambda_1,\dots,\lambda_\ell\in k[X_1,\dots,X_n]$ with $f=\lambda_1g_1+\cdots+\lambda_{\ell}g_{\ell}$. Suppose that $p\in \mathcal{V}(S)$, then $f(p)=\lambda_1g_1(p)+\cdots+\lambda_{\ell}g_{\ell}(p)=0$. Since all $p\in \mathcal{V}(S)$ is a zero of all $f\in \langle S\rangle$, we must have $\mathcal{V}(S)\subseteq \mathcal{V}(\langle S\rangle)$, and hence $\mathcal{V}(S)=\mathcal{V}(\langle S\rangle)$. The equality $\mathcal{V}(\langle S\rangle)=\mathcal{V}(\{f_1,\dots,f_m\})$ is derived similarly. Therefore, it is a common abuse of notation to write $\mathcal{V}(f_1,\dots,f_n)$ instead of $\mathcal{V}(\{f_1,\dots,f_n\})$. Conversely, we can define an ideal $\mathcal{I}(X)$ for each $X\subseteq \mathbf A^n(k)$. 

\begin{definition}Let $X\subseteq \mathbf A^n(k)$, then define the ideal $\mathcal{I}(X)$ of $k[X_1,\dots,X_n]$ as 
\[\mathcal{I}(X)=\{f\in k[X_1,\dots,X_n]:\forall p\in X,\,f(p)=0\}\]
which is a well-defined ideal as we can verify easily.
\end{definition}
In fact, not only is $\mathcal{I}(X)$ an ideal, it is also a radical ideal. We recall that an radical ideal of a commutative ring $R$ is an ideal $I\subseteq R$ with $I=\sqrt{I}$ where $\sqrt{I}=\{r\in R:\exists m>0,\, r^m\in I\}$. In other words, a radical ideal is an ideal $I\subseteq R$ where for all $r\in R$, if $r^m\in I$ for some $m>0$, then $r\in I$. To see that $\mathcal{I}(X)$ is a radical ideal, note that if $f^m\in \mathcal{I}(X)$ for some $m>0$, then $f^m(p)=0$ for all $p\in X$, then $f(p)=0$ for all $p\in X$ as $k$ is a field, hence $f\in \mathcal{I}(X)$. 

Similar to how $\mathcal{V}(S)\subseteq \mathcal{V}(T)$ when $T\subseteq S\subseteq k[X_1,\dots,X_n]$, we easily have $\mathcal{I}(X)\subseteq \mathcal{I}(Y)$ when $Y\subseteq X\subseteq \mathbf A^n(k)$. Let $f\in S\subseteq k[X_1,\dots,X_n]$, then by definition $f$ vanishes on all of $\mathcal{V}(S)$, therefore $f\in \mathcal{I}(\mathcal{V}(S))$. Let $p\in X\subseteq \mathbf A^n(k)$, then by definition $p$ is a zero of all polynomials of $\mathcal{I}(X)$, therefore $p\in \mathcal{V}(\mathcal{I}(X))$. Hence we have $S\subseteq \mathcal{I}(\mathcal{V}(S))$ and $X\subseteq \mathcal{V}(\mathcal{I}(X))$. From these facts, we derive that $\mathcal{I}(X)= \mathcal{I}(\mathcal{V}(\mathcal{I}(X)))$ and $\mathcal{V}(S)= \mathcal{V}(\mathcal{I}(\mathcal V(S)))$. In fact, we have the following.
\begin{theorembox}[Hilbert's Nullstellensatz]There is a bijective correspondance
\[\{\textrm{radical\ ideals\ of\ }k[X_1,\dots,X_n]\}\longleftrightarrow\{\textrm{algebraic\ sets\ of\ }\mathbf A^n(k)\}\]
given by $I\mapsto \mathcal{V}(I)$ and $X\mapsto \mathcal{I}(X)$.
\end{theorembox}
whose proof we will delay until later in this note. This observation is central to algebraic geometry. 

We observe that for a nonempty family of ideals $I_{\alpha}\subseteq k[X_1,\dots,X_n]$ indexed by $\alpha$, we have $\mathcal{V}(\sum_{\alpha}I_{\alpha})=\mathcal{V}(\bigcup_{\alpha}I_{\alpha})=\bigcap_{\alpha}\mathcal{V}(I_{\alpha})$. This should be easy to verify, and it tells us that the arbitrary intersection of algebraic sets is algebraic. Next, we observe that for ideals $I,J\subseteq k[X_1,\dots,X_n]$, we have $\mathcal{V}(I\cap J)=\mathcal{V}(I\cdot J)=\mathcal{V}(I)\cup \mathcal{V}(J)$, where the set $I\cdot J=\{fg:f\in I,g\in J\}$. Suppose that $p\in \mathcal{V}(I)\cup \mathcal{V}(J)$, assume without loss of generality that $p\in \mathcal{V}(I)$. For all $f\in I\cap J$, we have $f\in I$, so $f(p)=0$, hence $p\in \mathcal{V}(I\cap J)$, thus $\mathcal{V}(I)\cup\mathcal{V}(J)\subseteq \mathcal{V}(I\cap J)$. On the other hand, for each $fg\in I\cdot J$, we have $(fg)(p)=f(p)g(p)=0$, thus we have $\mathcal{V}(I)\cup\mathcal{V}(J)\subseteq \mathcal{V}(I\cdot J)$. Conversely, if $p\not\in \mathcal{V}(I)\cup \mathcal{V}(J)$, then there exists $f\in \mathcal{V}(I)$ and $g\in \mathcal{V}(J)$ such that $f(p)\ne 0$ and $g(p)\ne 0$, and hence $(fg)(p)=f(p)g(p)\ne 0$ as $k$ is a field. Since we know that $fg\in I\cdot J$ and $fg\in I\cap J$, we have $p\not\in \mathcal{V}(I\cdot J)$ and $p\not\in \mathcal{V}(I\cap J)$. Thus $\mathcal{V}(I\cap J),\mathcal{V}(I\cdot J)\subseteq \mathcal{V}(I)\cup\mathcal{V}(J)$, and hence we completed the proof. Moreover, since $IJ\subseteq I\cap J$, we have $\mathcal{V}(I)\cup\mathcal{V}(J)=\mathcal{V}(I\cap J)\subseteq\mathcal{V}(IJ)$, and since $I\cdot J\subseteq IJ$, we have $\mathcal{V}(IJ)\subseteq\mathcal{V}(I\cdot J)=\mathcal{V}(I)\cup\mathcal{V}(J)$. Hence $\mathcal{V}(IJ)=\mathcal{V}(I)\cup\mathcal{V}(J)$ as well. From these facts, we conclude that the complements of algebraic sets form a topology of the affine space. This topology is known as the \textit{Zariski topology}, and the Zariski closed sets are just the algebraic sets.


\begin{definitionbox}An algebraic set $V$ is \textit{irreducible} if it cannot be written as $V=V_1\cup V_2$ where the algebraic sets $V_1,V_2\subset V$ properly, and such a set is called an \textit{(algebraic) variety}.
\end{definitionbox}
\begin{exercise}If $\mathcal{I}(X)=\mathcal{I}(Y)$ for algebraic sets $X,Y$, then $X=Y$.
\end{exercise}
\begin{lemma}An algebraic set $V$ is a variety iff $\mathcal{I}(V)$ is prime.
\begin{proof}Suppose that $\mathcal{I}(V)$ is not prime, that $fg\in \mathcal{I}(V)$ and $f,g\not\in \mathcal{I}(V)$. We claim that 
\[V=(V\cap \mathcal{V}(f))\cup (V\cap \mathcal{V}(g))\]
Let $p\in V$, then $(fg)(p)=f(p)g(p)=0$, thus $f(p)=0$ or $g(p)=0$ since $k$ is a field. Hence we have $p\in \mathcal{V}(f)$ or $p\in \mathcal{V}(g)$. Therefore $V\subseteq (V\cap \mathcal{V}(f))\cup (V\cap \mathcal{V}(g))$, the other direction $(V\cap \mathcal{V}(f))\cup (V\cap \mathcal{V}(g))\subseteq V$ is obvious. Since $f\not\in \mathcal{I}(V)$, exists $p\in V$ with $f(p)\ne 0$. Thus $p\not\in \mathcal{V}(f)$. Thus $V\ne V\cap \mathcal{V}(f)$. Similarly, $V\ne V\cap \mathcal{V}(g)$, so $V$ is reducible. Conversely, assume $V=V_1\cup V_2$ where $V_1,V_2\subset V$ properly. We have $\mathcal{I}(V)\subset \mathcal{I}(V_1),\mathcal{I}(V_2)$ properly. Choose $f\in \mathcal{I}(V_1)\setminus \mathcal{I}(V)$ and $g\in \mathcal{I}(V_2)\setminus \mathcal{I}(V)$. For $p\in V$, we have $p\in V_1$ or $p\in V_2$, thus $f(p)=0$ or $g(p)=0$, so $(fg)(p)=f(p)g(p)=0$. Hence $fg\in \mathcal{I}(V)$, so $\mathcal{I}(V)$ is not prime.
\end{proof}
\end{lemma}


Take, for example, the algebraic set $V=\mathcal{V}(f,g)\subseteq \mathbf A^3(\mathbf R)$ where $f(x,y,z)=x^2+y^2+z^2-4$ and $g(x,y,z)=y^2+z^2-1$. Then $V$ is the intersection of the sphere of radius $2$, and the cylinder of radius $1$. In fact, we have a decomposition of $V$
\[V=\mathcal{V}(x-\sqrt{3},y^2+z^2-1)\cup \mathcal{V}(x+\sqrt{3},y^2+z^2-1)\]
into algebraic varieties. This is easy to visualize and check that it is true. In fact, we can do even better. We will show that each algebaic set has a unique decomposition into algebraic varieties. Suppose that $R$ is a commutative ring, we recall the following theorem.
\begin{theorem}\label{thm:min}$R$ is noetherian iff every nonempty set of ideals has a maximal element.
\end{theorem}
\begin{theorem}If $V$ is an algebraic set, then $V$ has a unique decomposition $V=V_1\cup\cdots\cup V_m$, where $V_1,\dots,V_m$ are varieties such that no one of them is contained in another.
\begin{proof}
Suppose that $\mathcal{L}$ is the set of algebraic sets that do not admit a finite variety decomposition, we will show that $\mathcal{L}=\emptyset$. Suppose the contrary, then $\mathcal{L}$ has a minimal element $V$ w.r.t inclusion by \autoref{thm:min} on $\mathcal{I}[\mathcal{L}]$. Since $V\in A$, we have $V$ is reducible, hence $V=V_1\cup V_2$ with algebraics sets $V_1,V_2\subset V$ properly. Since $V$ is minimal, we must have $V_1,V_2\not\in \mathcal{L}$. Thus $V_1,V_2$ admit finite variety decompositions, contradiction. Next, we show the uniqueness. Let $V=V_1\cup\cdots\cup V_m=W_1\cup\dots\cup W_h$ be decompositions, then $V_i=(V_i\cap W_1)\cup\cdots\cup (V_i\cap W_h)$, which by the irreducibility of $V_i$, tells us that $V_i\subseteq W_{\sigma(i)}$ for some $\sigma(i)$. Similarly $W_j\subseteq V_{\delta(j)}$ for some $\delta(j)$. Thus $V_i\subseteq W_{\sigma(i)}\subseteq V_{\delta(\sigma(i))}$. However, $V_i\subseteq V_{\delta(\sigma(i))}$ implies that $V_i=V_{\delta(\sigma(i))}$, so $i=\delta(\sigma(i))$ and $V_i=W_{\sigma(i)}$. 
\end{proof}
\end{theorem}

\begin{proposition}Suppose that $k$ is algebraically closed and $F\in k[X,Y]$ is a nonconstant with decomposition $F=F_1^{n_1}\cdots F_r^{n_r}$ into irreducible polynomials, then $\mathcal{V}(F)=\mathcal{V}(F_1)\cup\cdots\cup \mathcal{V}(F_r)$ is the decomposition of $\mathcal{V}(F)$ into varieties, and $\mathcal{I}(\mathcal{V}(F))=(F_1\cdots F_r)$.
\end{proposition}

\section{Hilbert's Nullstellensatz}

\chapter{Schemes}
\section{Presheaves and Sheaves}
Let $X$ be a topological space, and $\mathcal C$ a category whose objects are ``sets with extra structures'' and morphisms are ``functions with structure-preserving properties'', i.e. a ``concrete category''. You may think of $\mathcal C$ as one of the categories $\mathbf{Set},\mathbf{Ab},\mathbf{CRing}$, or $\mathbf {Mod}_R$.
\begin{definition}\label{def:presheaf}A \textit{presheaf} $\mathscr{F}$ of $\mathcal C$ on $X$ is the data of the following
\begin{enumerate}[label=(\roman*)]
	\item for each open $U\subseteq X$, some $\mathscr{F}(U)\in \mathcal C$, whose elements are called the \textit{sections} of $\mathscr{F}$ over $U$,
	\item for each inclusion $U\hookrightarrow V$ of open sets of $X$, a \textit{restriction} map $\mathrm{Res}^V_U:\mathscr{F}(V)\rightarrow\mathscr{F}(U)$, s.t.
	\begin{itemize}
		\item for all open $U\subseteq X$, the map $\mathrm{Res}^U_U=\mathbf 1_{\mathscr{F}(U)}$,
		\item if $U\hookrightarrow V \hookrightarrow W$ are inclusions of open sets of $X$ then the diagram
		\[
		\begin{tikzcd}[sep=huge]
			\mathscr{F}(W) \arrow[r,"\mathrm{Res}^W_V"] \arrow[rr, "\mathrm{Res}^W_U", bend left, shift left=1] & \mathscr{F}(V) \arrow[r,"\mathrm{Res}^V_U"] & \mathscr{F}(U)
		\end{tikzcd}
		\]
		commutes, or in other words $\mathrm{Res}^W_U=\mathrm{Res}^V_U\circ \mathrm{Res}^W_V$
	\end{itemize}
\end{enumerate}
Moreover, suppose that $\mathscr{F},\mathscr{G}$ are presheaves of $\mathcal C$ on $X$, then a morphism $\phi:\mathscr{F}\rightarrow\mathscr{G}$ is the data of a map $\phi(U):\mathscr{F}(U)\rightarrow\mathscr{G}(U)$ for each open $U\subseteq X$ such that whenever $U\hookrightarrow V$ is an inclusion of open sets, the following diagram (where we abuse the notation $\mathrm{Res}^V_U$)
\[
\begin{tikzcd}[sep = huge]
\mathscr{F}(V) \arrow[r, "\phi(V)"] \arrow[d, "\mathrm{Res}^V_U"'] & \mathscr{G}(V) \arrow[d, "\mathrm{Res}^V_U"] \\
\mathscr{F}(U) \arrow[r, "\phi(U)"]                 & \mathscr{G}(U)               
\end{tikzcd}\]
commutes, i.e. $\mathrm{Res}^V_U\circ\phi(V)=\phi(U)\circ\mathrm{Res}^V_U$. The composition of these morphisms is element-wise.
\end{definition}
In fact, let $\mathcal{G}_X$ be the category of open sets of $X$, i.e. the objects of $\mathcal{G}_X$ are the open sets of $X$ and the morphisms are the inclusions of open sets, then a presheaf of $\mathcal C$ on $X$ is the same information as a contravariant functor from $\mathcal{G}_X$ to $\mathcal C$. Moreover, a morphism between presheaves of $\mathcal C$ on $X$ is the same information as a natural transformation between contravariant functors from $\mathcal{G}_X$ to $\mathcal C$. Thus the category of presheaves of $\mathcal C$ on $X$ is just the contravariant functor category $\mathrm{Func}^{\mathrm{op}}(\mathcal{G}_X,\mathcal C)$. One can check that this is just a restatement of \autoref{def:presheaf}. Therefore, we will view a presheaf of $\mathcal C$ on $X$ just as a contravariant functor from $\mathcal{G}_X$ to $\mathcal C$. 
\begin{definition}A presheaf $\mathscr{F}$ of $\mathcal C$ on $X$ is a \textit{sheaf} if for any open set $U\subseteq X$ and any open cover $\{U_i\}_{i\in\mathcal{I}}$ of $U$ the following two axioms are satisfied:
\begin{enumerate}[label=(\roman*)]
	\item \textit{identity axiom}: if $f,g\in\mathscr{F}(U)$, then $\mathrm{Res}^U_{U_i}(f)=\mathrm{Res}^U_{U_i}(g)$ for all $i\in\mathcal{I}$ implies that $f=g$, 
	\item \textit{gluability axiom}: if $f_i\in \mathscr{F}(U_i)$ for all $i\in\mathcal{I}$ is such that $\mathrm{Res}^{U_i}_{U_i\cap U_j}(f_i)=\mathrm{Res}^{U_j}_{U_i\cap U_j}(f_j)$ for all $i,j\in\mathcal{I}$, then there exists some $f\in\mathscr{F}(U)$ such that $\mathrm{Res}^U_{U_i}(f)=f_i$ for all $i\in\mathcal{I}$.
\end{enumerate}
Moreover, we will write a \textit{(pre)sheaf of sets} on $X$ to mean a (pre)sheaf of $\mathbf{Set}$ on $X$, and a \textit{(pre)sheaf of rings} on $X$ to mean a (pre)sheaf of $\mathbf{CRing}$ on $X$, and so on for all other possibilities for $\mathcal C$.
\end{definition}
\begin{example}A motivating example is the following. Suppose that $X=\mathbf R^n$, then define a sheaf of rings $\mathscr{F}$ on $X$ by letting $\mathscr{F}(U)=\{\mathrm{smooth\ functions\ }U\rightarrow \mathbf R\}$ for all open $U\subseteq X$ with the natural operations. For an inclusion of open sets $U\hookrightarrow V$ the restriction map $\mathrm{Res}^V_U$ is defined in the obvious way. We can easily check that $\mathscr{F}$ is well-defined. This sheaf is the \textit{sheaf of rings of smooth functions}.
\end{example}
\begin{definition}Let $\mathscr{F}$ be a presheaf of $\mathcal C$ on $X$, the \textit{stalk} $\mathscr{F}_p$ of $\mathscr{F}$ at a point $p\in X$, whose elements are called \textit{germs} of $\mathscr{F}$ at $p$, is an object of $\mathcal C$ where
\[\mathscr{F}_p=\left.\left\{(f,U):\begin{array}{c}f\in\mathscr{F}(U),\textrm{\ and\ }U\subseteq X\textrm{\ is}\\\textrm{an\ open\ neighborhood\ of\ }p\end{array}\right\} \middle/ \left\{\begin{array}{c}(f,U)\sim (g,V)\textrm{\ if\ and\ only\ if\ exists\ open}\\W\subseteq U\cap V\textrm{\ such\ that\ }\mathrm{Res}^U_W(f)=\mathrm{Res}^V_W(g)\end{array}\right\}\right.\]
with its extra structure inherited from $\mathscr{F}$ in the obvious way. For example, if $\mathcal C=\mathbf{CRing}$, then the extra structure of $\mathscr{F}_p$, the addition and multiplication, is defined as follows,
\[\begin{aligned} (f,U)+(g,V)&=(\mathrm{Res}^U_{U\cap V}(f)+\mathrm{Res}^V_{U\cap V}(g),U\cap V)\\ (f,U)(g,V)&=(\mathrm{Res}^U_{U\cap V}(f)\mathrm{Res}^V_{U\cap V}(g),U\cap V)\end{aligned}\]
We can check that this is well defined. In fact, we can formalize $\mathscr{F}_p$ as a colimit
\[\mathscr{F}_p=\lim_{\longrightarrow}\,\mathscr{F}(U)\textrm{\ indexed\ over\ }p\in U\in\mathcal G_X\]

\end{definition}

\end{document}
