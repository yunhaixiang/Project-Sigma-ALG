\RequirePackage{silence}
\WarningFilter{remreset}{The remreset package}
\documentclass[11pt]{book}
\usepackage{amsmath,amsthm,amssymb}
\usepackage[mathscr]{eucal}
\usepackage[hidelinks]{hyperref}
\usepackage[margin=1in]{geometry}
\usepackage{standalone}
\usepackage{enumitem}
\usepackage{fancyhdr}
\usepackage{setspace}
\usepackage{tikz}
\usepackage{mathpazo}
\usepackage{thmtools}
\usepackage{tikz}
\usepackage{tikz-cd}
\usepackage{xcolor}
\hypersetup{
    colorlinks,
    linkcolor={red!50!black},
    citecolor={blue!50!black},
    urlcolor={blue!80!black}
}
%\usepackage{tcolorbox}

%%%%%%%%%%%%% HEADERS %%%%%%%%%%%%%%%
\pagestyle{fancy}
\fancyhf{}
\rhead{\thepage}
\lhead{\leftmark}
\headheight=13.6pt

%%%%%%%%%%%%% ENVIRONMENTS %%%%%%%%%%%%%%%
\declaretheoremstyle[headfont=\color{blue!45!black!}\normalfont\bfseries]{bluestyle}
\declaretheoremstyle[headfont=\color{green!45!black!}\normalfont\bfseries]{greenstyle}
\declaretheoremstyle[headfont=\color{red!45!black!}\normalfont\bfseries]{redstyle}
\declaretheorem[numberwithin=section, style=definition]{definition}
\declaretheorem[sharenumber=definition, style=definition]{theorem}
\declaretheorem[sharenumber=definition, style=definition]{exercise}
\declaretheorem[sharenumber=definition, style=definition]{lemma}
\declaretheorem[sharenumber=definition, style=definition]{proposition}
\declaretheorem[sharenumber=definition, style=definition]{corollary}
\declaretheorem[sharenumber=definition, style=definition]{remark}
\declaretheorem[sharenumber=definition, style=definition]{axiom}
\declaretheorem[sharenumber=definition, style=definition]{example}
\declaretheorem[sharenumber=definition, style=definition]{result}
\declaretheorem[sharenumber=definition, style=definition]{problem}
\declaretheorem[sharenumber=definition, style=definition]{conjecture}
\declaretheorem[sharenumber=definition, style=definition]{algorithm}
\declaretheorem[sharenumber=definition, style=definition]{heuristic}
\declaretheorem[sharenumber=definition, style=definition]{motivation}
\declaretheorem[sharenumber=definition, style=definition]{intuition}
\declaretheorem[sharenumber=definition, style=definition]{anecdote}
\declaretheorem[sharenumber=definition, style=definition]{abbriviation}
\declaretheorem[sharenumber=definition, style=definition]{convention}

\declaretheorem[name={Definition},sharenumber=definition, shaded={margin=0.025\linewidth, textwidth=0.95\linewidth, bgcolor={red!3!white!}}, style=redstyle]{definitionbox}

\declaretheorem[name={Theorem},sharenumber=definition, shaded={margin=0.025\linewidth, textwidth=0.95\linewidth, bgcolor={blue!3!white!}}, style=bluestyle]{theorembox}

\declaretheorem[name={Exercise},sharenumber=definition, shaded={margin=0.025\linewidth, textwidth=0.95\linewidth, bgcolor={green!3!white!}}, style=greenstyle]{exercisebox}
\newenvironment{solution}{\begin{proof}[Solution]}{\end{proof}}
\newenvironment{abc}{\begin{enumerate}[label=(\alph*)]}{\end{enumerate}}
\renewcommand{\sectionautorefname}{\S}
\renewcommand{\subsectionautorefname}{\S}
\renewcommand{\subsubsectionautorefname}{\S}
\renewcommand{\thetable}{\thechapter.\Alph{table}}

%%%%%%%%%%%%% SHORTCUTS %%%%%%%%%%%%%%%
\DeclareMathOperator{\N}{\mathbf{N}}
\DeclareMathOperator{\Z}{\mathbf{Z}}
\DeclareMathOperator{\R}{\mathbf{R}}
\DeclareMathOperator{\F}{\mathbf{F}}
\DeclareMathOperator{\Q}{\mathbf{Q}}
\DeclareMathOperator{\GL}{\mathrm{GL}}
\DeclareMathOperator{\SL}{\mathrm{SL}}
\DeclareMathOperator{\lcm}{\operatorname{lcm}}
\DeclareMathOperator{\ord}{\operatorname{ord}}
\DeclareMathOperator{\sgn}{\operatorname{sgn}}
\DeclareMathOperator{\Aut}{\operatorname{Aut}}
\DeclareMathOperator{\Inn}{\operatorname{Inn}}
\DeclareMathOperator{\End}{\operatorname{End}}
\DeclareMathOperator{\stab}{\operatorname{stab}}
\DeclareMathOperator{\orb}{\operatorname{orb}}
%\DeclareMathOperator{\ker}{\operatorname{ker}}
\DeclareMathOperator{\im}{\operatorname{im}}
\DeclareMathOperator{\IM}{\operatorname{Im}}
\DeclareMathOperator{\RE}{\operatorname{Re}}
\DeclareMathOperator{\Span}{\operatorname{span}}
\DeclareMathOperator{\spec}{\operatorname{spec}}
\DeclareMathOperator{\Char}{\operatorname{char}}
\DeclareMathOperator{\Rank}{\operatorname{rank}}
\DeclareMathOperator{\proj}{\operatorname{proj}}
\DeclareMathOperator{\normal}{\trianglelefteq}


\title{
\vspace{-2.0cm}
\Large{Project Sigma}\\
\vspace{1cm}
\huge{\bf{Algebraic Geometry}}
\vspace{3cm}}
\author{Yunhai Xiang}
\date{\today}

\begin{document}


\maketitle
\doublespacing
\tableofcontents
\singlespacing
\newpage
%\setlength{\parskip}{10pt}
\chapter{Preliminaries}

\section{Categories, Functors, and Natural Transformations}
To discuss algebraic geometry cleanly and efficiently, we need to introduce category theory. Given sets $X$ and $Y$, from elementary set theory, a \textit{function} $f:X\rightarrow Y$ consists of the information of its \textit{domain} $X$, \textit{codomain} $Y$, and \textit{graph} $G\subseteq X\times Y$ where for each $x\in X$ there exists unique $y\in Y$ such that $(x,y)\in G$. In category theory, we generalize the notion of a function. Suppose we are given two \textit{objects} $X$ and $Y$, a \textit{morphism} $f:X\rightarrow Y$ consists of the information of its \textit{source} $X$, \textit{target} $Y$, and possibly some additional information. It is meaningless to talk about a morphism without talking about the category it belongs to, therefore we will make these terminologies precise.
\begin{definition}A \textit{category} $\mathcal C$ is the data of
\begin{enumerate}[label=(\roman*)]
	\item a class, also denoted $\mathcal C$, whose elements are called \textit{objects} of $\mathcal C$,
	\item for each pair of objects $X,Y\in \mathcal C$, a class $\mathrm{Mor}(X,Y)$ of \textit{morphisms} from $X$ to $Y$, such that each $f\in \mathrm{Mor}(X,Y)$ consists of the data of its \textit{source} $X$, \textit{target} $Y$, and possibly additional data,
	\item for each triple of object $X,Y,Z\in\mathcal C$, a \textit{composition} function 
	\[(-\circ -):\mathrm{Mor}(Y,Z)\times\mathrm{Mor}(X,Y)\rightarrow \mathrm{Mor}(X,Z)\]
	such that the following axioms are satisfied
	\begin{itemize}
		\item \textit{identity axiom}: for each $X\in\mathcal{C}$, exists $\mathbf 1_X\in\mathrm{Mor}(X,X)$ s.t. if $Y\in\mathcal C$, for all $f\in\mathrm{Mor}(X,Y)$ we have $f\circ \mathbf 1_X=f$ and for all $g\in \mathrm{Mor}(Y,X)$ we have $\mathbf 1_X\circ g=g$.
		\item \textit{associativity axiom}: for each quadruple of objects $X,Y,Z,W\in \mathcal C$, given $f\in \mathrm{Mor}(X,Y)$, $g\in \mathrm{Mor}(Y,Z)$, and $h\in\mathrm{Mor}(Z,W)$, we have $(h\circ g)\circ f=h\circ (g\circ f)$.
	\end{itemize}
\end{enumerate}
Moreover, we say that the category is \textit{locally small} if $\mathrm{Mor}(X,Y)$ is a set for all objects $X,Y$, and \textit{small} if it is locally small and the class of objects is also a set. By convention, we use $f:X\rightarrow Y$ to denote $f\in\mathrm{Mor}(X,Y)$ when there is no risk of confusion. 
\end{definition}
\begin{example}The quintessential example to keep in mind is the category $\mathbf{Set}$ whose objects are sets and the morphisms are just functions between sets with the usual composition. In fact, many categories we encounter are such that their objects are ``sets with extra structures'' and morphisms are ``functions with structure-preserving properties''. They are known as \textit{concrete categories} but we will not make this term precise or use this terminology. In the next page, we will provide a list of the definitions of these type of categories.
\end{example}

\begin{table}[ht]
\centering
\caption{Table of some useful categories}
\vspace{3mm}
\begin{tabular}[t]{l|c|c|c}
\textit{name of category} & \textit{notation} & \textit{objects} & \textit{morphisms}\\
\hline
category of sets &$\mathbf{Set}$ & sets & functions\\
category of groups & $\mathbf{Grp}$ & groups & group homomorphisms\\
category of abelian groups & $\mathbf{Ab}$ & abelian groups & group homomorphisms \\
category of rings & $\mathbf{CRing}$ & (commutative) rings & ring homomorphisms\\
category of posets & $\mathbf{Pos}$ & posets & increasing functions\\
category of graphs & $\mathbf{Graph}$ & (simple) graphs & graph homomorphisms\\
category of topological spaces & $\mathbf{Top}$ & topological spaces & continuous functions\\
category of Hausdorff spaces & $\mathbf{Haus}$ & Hausdorff spaces & continuous functions\\
category of $\mathscr{C}^\ell$ manifolds & $\mathbf{Man}_{\ell}$ & $\mathscr{C}^\ell$ manifolds & $\mathscr{C}^\ell$ maps\\
category of affine spaces over $k$ & $\mathbf{Aff}_k$ & affine spaces over $k$ & affine maps\\
category of vector spaces over $k$ & $\mathbf{Vect}_k$ & vector spaces over $k$ & linear maps\\
category of modules over $R$ & $\mathbf{Mod}_R$ & modules over $R$ & module homomorphisms\\
\end{tabular}
\label{tab:cat}
\end{table}
\begin{remark}Throughout this note, the term rings refers to commutative rings (with unit), thus we will not introduce the category $\mathbf{Ring}$ which includes possibly non-commutative rings.
\end{remark}
\begin{definition}Suppose $\mathcal C$ is a category with $X,Y\in\mathcal C$, a morphism $f:X\rightarrow Y$ is a(n)
\begin{enumerate}[label=(\roman*)]
	\item \textit{endomorphism} if $X=Y$,
	\item \textit{monomorphism} if left-cancellable, i.e. $f\circ g=f\circ h$ implies $g=h$ for $g,h:Z\rightarrow X$ and $Z\in\mathcal C$,
	\item \textit{epimorphism} if right-cancellable, i.e. $g\circ f=h\circ f$ implies $g=h$ for $g,h:X\rightarrow Z$ and $Z\in\mathcal C$,
	\item \textit{split monomorphism} if it admits left-inverse, i.e. there exists $g:Y\rightarrow X$ such that $g\circ f=\mathbf 1_X$,
	\item \textit{split epimorphism} if it admits right-inverse, i.e there exists $g:Y\rightarrow X$ such that $f\circ g=\mathbf 1_Y$,
	\item \textit{bimorphism} if it's both a monomorphism and an epimorphism,
	\item \textit{isomorphism} if it's both a split monomorphism and a split epimorphism,
	\item \textit{automorphism} if it's both an isomorphism and an endomorphism. 
\end{enumerate}
We say $X$ and $Y$ are \textit{isomorphic} and write $X\cong Y$ if exists isomorphism $f:X\rightarrow Y$. A \textit{groupoid} is a category where all morphisms are isomorphisms. Denote $\mathrm{End}(X)=\mathrm{Mor}(X,X)$ for all $X\in\mathcal C$.
\end{definition}
In $\mathbf{Set}$, the monomorphisms are exactly the injections and epimorphisms exactly surjections. In ``concrete categories'', the injective morphisms are monic and the surjective morphisms are epic, but the converes need not hold. A classical counterexample is the canonical inclusion $\mathbf Z\hookrightarrow\mathbf Q$ in $\mathbf{CRing}$, which is epic but is clearly not surjective. This is also an example of an epimorphism that is not split. Straightforwardly, split monomorphisms are monomorhpisms and split epimorphisms are epimorphisms. The converses need not hold as we have seen in the previous example. Hence, a bimorphism need not be an isomorphism, even though an isomorphism must be a bimorphism. 

\begin{example}We should note that a morphism need not be a function. A classical example is the category $\mathbf{Rel}$ where the objects are sets and each morphism $f:X\rightarrow Y$ is just a subset $f\subseteq X\times Y$. Given two morphisms $f:X\rightarrow Y$ and $g:Y\rightarrow Z$, then the composition $g\circ f:X\rightarrow Z$ is the set $g\circ f=\{(a,c)\in X\times Z:\exists b\in Y,(a,b)\in f\textrm{\ and\ }(b,c)\in g\}\subseteq X\times Z$. We can check easily that $\mathbf{Rel}$ is well defined, and the identity morphism on each object is the equality relation.
\end{example}
\begin{definition}Let $\mathcal C$ be a category, then a \textit{subcategory} of $\mathcal C$ is a category $\mathcal D$ such that their classes of objects $\mathcal D\subseteq\mathcal C$ and $\mathrm{Mor}_{\mathcal D}(X,Y)\subseteq \mathrm{Mor}_{\mathcal C}(X,Y)$ for all $X,Y\in\mathcal D$, and the composition of $\mathcal D$ is inherited from $\mathcal C$. Note that $\mathcal D$ is well-defined iff it is closed under composition and it contains the identity morphisms. Moreover, we say $\mathcal D$ is a \textit{full subcategory} if $\mathrm{Mor}_{\mathcal D}(X,Y)=\mathrm{Mor}_{\mathcal C}(X,Y)$ for $X,Y\in\mathcal D$, an \textit{isomorphism-closed subcategory} if for all $X\in\mathcal D$ and all $Y\in \mathcal C$ isomorphic to $X$ in $\mathcal C$ we have $Y$ and all isomorphisms between $X$ and $Y$ are contained in $\mathcal D$, and a \textit{strictly full subcategory} if it is both a full subcategory and an isomorphism-closed subcategory.
\end{definition}
\begin{definition}Let $\mathcal C,\mathcal D$ be categories, define the \textit{product category} $\mathcal C\times\mathcal D$ as the category where objects are ordered pairs $\langle X,Y\rangle$ where $X\in\mathcal C$ and $Y\in\mathcal D$, and a morphism $f:\langle X,Y\rangle\rightarrow \langle U,V\rangle$ is a pair $\langle g,h\rangle$ where $g\in\mathrm{Mor}_{\mathcal C}(X,U)$ and $h\in \mathrm{Mor}_{\mathcal D}(Y,V)$, with composition component-wise.
\end{definition}
\begin{definition}Let $\mathcal C$ and $\mathcal D$ be categories, a \textit{(covariant) functor} (resp. \textit{contravariant functor}) $\mathscr{F}$ from $\mathcal C$ to $\mathcal D$, denoted $\mathscr{F}:\mathcal C\rightarrow\mathcal{D}$, is the data of its \textit{source} $\mathcal C$, \textit{target} $\mathcal D$, and 
\begin{enumerate}[label=(\roman*)]
	\item for each object $X\in\mathcal C$, an object $\mathscr{F}(X)\in\mathcal D$,
	\item for each pair of objects $X,Y\in\mathcal C$ and each $f:X\rightarrow Y$ of $\mathcal C$, a morphism $\mathscr{F}f:\mathscr{F}(X)\rightarrow\mathscr{F}(Y)$ (resp. a morphism $\mathscr{F}f:\mathscr{F}(Y)\rightarrow\mathscr{F}(X)$) of $\mathcal D$ such that 
	\begin{itemize}
		\item $\mathscr{F}\mathbf{1}_X=\mathbf 1_{\mathscr{F}(X)}$ for all $X\in\mathcal C$, and
		\item $\mathscr{F}(g\circ f)=\mathscr{F}g\circ\mathscr{F}f$ (resp. $\mathscr{F}(g\circ f)=\mathscr{F}f\circ \mathscr{F}g$) for all $f:X\rightarrow Y$ and $g:Y\rightarrow Z$.
	\end{itemize}
\end{enumerate}
The composition of covariant functors (resp. contravariant functors) is element-wise. Equivalently, a contravariant functor $\mathscr{F}:\mathcal C\rightarrow\mathcal D$ can be viewed as a covariant functor $\mathscr{F}:\mathcal C^{\mathrm{op}}\rightarrow\mathcal D$, where $\mathcal C^{\mathrm{op}}$ is the \textit{opposite category} of $\mathcal C$, i.e. the category with the same objects as $\mathcal C$, and $\mathrm{Mor}_{\mathcal{C}^{\mathrm{op}}}(X,Y)$ is the class $\mathrm{Mor}_{\mathcal C}(Y,X)$ with each of its element's source and target swapped, for all $X,Y\in\mathcal C^{\mathrm{op}}$. Needless to say, the composition of $\mathcal{C}^{\mathrm{op}}$ is defined symmetric to that of $\mathcal C$.
\end{definition}
The reason we use the notation $\mathscr{F}:\mathcal C\rightarrow\mathcal D$ is that we view functors as ``morphisms'' between categories, but there is just this small issue: there can not possibly be a category of all categories, as that would lead to set-theoretic issues. We can, however, define the category $\mathbf{Cat}$ whose objects are all small categories with functors as morphisms. Note that even if one of $\mathcal C$ or $\mathcal D$ is not small, we can still talk about functors between them and and use the notation $\mathscr{F}:\mathcal C\rightarrow\mathcal D$ even if technically $\mathscr{F}$ is not a morphism of any category we defined. 
\begin{example}For each category $\mathcal C$, let $\mathbf 1_{\mathcal C}:\mathcal C\rightarrow\mathcal C$ denote the \textit{identity functor}, which is the functor that maps all objects and all morphisms to themselves. With a slight abuse of notation, we also let $\mathbf 1_{\mathcal C}:\mathcal C\rightarrow\mathcal C^{\mathrm{op}}$ denote the \textit{contravariant identity functor}, which is the contravariant functor that maps each object to itself and swaps the source and target of each morphism.
\end{example}
\begin{example}Define $\mathbf{Top}_{\bullet}$ as the category of pointed topological spaces, i.e. the category where the objects are topological spaces with a choice of a base point and the morphisms are continuous functions that preserve the base points. Define functor $\pi_1:\mathbf{Top}_{\bullet}\rightarrow\mathbf{Grp}$ which assigns to each pointed topological space $\langle x,X\rangle$ the fundamental group $\pi_1(x,X)$. For each $f:\langle x,X\rangle\rightarrow \langle y,Y\rangle$, we assign $\pi_1f:\pi_1(x,X)\rightarrow\pi_1(y,Y)$ the induced homomorphism $[\varphi]\mapsto [f\circ\varphi]$.
\end{example}
\begin{example}Suppose $\mathcal C$ is a locally small category with $A\in\mathcal C$. Let the \textit{(covariant) hom-functor} $\mathscr{H}_A:\mathcal C\rightarrow \mathbf{Set}$ be $\mathscr{H}_A(X)=\mathrm{Mor}_{\mathcal C}(A,X)$ for each $X\in\mathcal C$ and $\mathscr{H}_Af=(f\circ -)$ for each $f:X\rightarrow Y$, where the function $(f\circ -):\mathrm{Mor}_{\mathcal C}(A,X)\rightarrow \mathrm{Mor}_{\mathcal C}(A,Y)$ is given by $g\mapsto f\circ g$. The \textit{contravariant hom-functor} $\mathscr{H}^A:\mathcal C\rightarrow \mathbf{Set}$ is defined by $\mathscr{H}^A(X)=\mathrm{Mor}_{\mathcal C}(X,A)$ for each $X\in\mathcal C$ and $\mathscr{H}^Af=(-\circ f)$ for each $f:X\rightarrow Y$, where $(-\circ f):\mathrm{Mor}_{\mathcal C}(Y,A)\rightarrow \mathrm{Mor}_{\mathcal C}(X,A)$ is given by $g\mapsto g\circ f$.
\end{example}
\begin{example}A \textit{(combinatorial) species} is a functor $\mathscr{F}:\mathbf{Fin}\rightarrow \mathbf{Fin}$ where $\mathbf{Fin}$ is the \textit{groupoid of finite sets}, the category where objects are finite sets and morphisms are bijections. Usually, $\mathscr{F}(X)$ is a set of labelled structures built from $X$. This is an extremely powerful tool in enumeration, since we can assign each species $\mathscr{F}$ the exponential generating funciton 
$\#{\mathscr{F}}(X)=\sum_{n\in\mathbf N}|\mathscr{F}(\{1,\dots,n\})|\frac{X^n}{n!}\in\mathbf{Q}[X]$
which encodes all information of $\mathscr{F}$.
\end{example}
\begin{example}A \textit{forgetful functor} is a functor that ``forgets'' certain structures or certain properties, e.g. the functor $\mathbf{Grp}\rightarrow\mathbf{Set}$ that assigns each group to its underlying set, the functor $\mathbf{Ab}\rightarrow \mathbf{Grp}$ that assigns each abelian group itself, and the functor $\mathbf{Mod}_R\rightarrow \mathbf{Ab}$ that assigns each module its additive group structure. Here, the term ``forgetful functor'' is used non-rigorously.
\end{example}
\begin{definition}We say that a covariant or contravariant functor $\mathscr{F}:\mathcal C\rightarrow \mathcal D$ is \textit{faithful} (resp. \textit{full}) if the function 
$\mathrm{Mor}_{\mathcal C}(X,Y)\rightarrow\mathrm{Mor}_{\mathcal D}(\mathscr F(X),\mathscr F(Y))$ or $\mathrm{Mor}_{\mathcal C}(X,Y)\rightarrow\mathrm{Mor}_{\mathcal D}(\mathscr F(Y),\mathscr F(X))$ given by $f\mapsto \mathscr{F}f$ is injective (resp. surjective) for all $X,Y\in\mathcal C$. We say that $\mathscr{F}$ is \textit{fully faithful} if it's both full and faithful. We say that $\mathscr{F}$ is \textit{essentially surjective} if for each $Y\in\mathcal D$ exists $X\in\mathcal C$ with $\mathscr{F}(X)\cong Y$, \textit{conservative} if $f:X\rightarrow Y$ is an isomorphism whenever $\mathscr Ff$ is.
\end{definition}
\begin{definition}Suppose $\mathcal I$ is a small category. A \textit{diagram} of \textit{shape} (or \textit{indexed over}) $\mathcal I$ in a category $\mathcal C$ is a functor $\mathscr{F}:\mathcal I\rightarrow \mathcal C$. We say $\mathscr{F}$ \textit{commutes} if $\mathscr Ff=\mathscr{F}g$ for all $f,g\in\mathrm{Mor}_{\mathcal I}(X,Y)$ where $X,Y\in\mathcal I$.
\end{definition}
\begin{example}Some example of shapes of diagrams are
\[
\begin{tikzcd}[sep = large]
\bullet \arrow[d] \arrow[rd] &   \\
\bullet \arrow[r]                 & \bullet
\end{tikzcd}
\qquad
\begin{tikzcd}[sep = large]
\bullet \arrow[r] \arrow[d] & \bullet \arrow[d] \\
\bullet \arrow[r]                 & \bullet         
\end{tikzcd}
\qquad
\begin{tikzcd}[sep = huge]
\bullet \arrow[r, bend left] \arrow[r, bend right] & \bullet \arrow[l]
\end{tikzcd}
\]
where, by convention, each bullet represents a distinct object and each arrow a distinct morphism, and all identity morphisms are omitted. A diagram in each of these shapes looks like
\[
\begin{tikzcd}[sep = large]
A \arrow[d, "h"'] \arrow[rd, "f"] &   \\
B \arrow[r, "g"']                 & C
\end{tikzcd}
\qquad
\begin{tikzcd}[sep = large]
A \arrow[r, "f"] \arrow[d, "g"'] & B \arrow[d, "h"] \\
C \arrow[r, "i"]                 & D         
\end{tikzcd}
\qquad
\begin{tikzcd}[sep = huge]
A \arrow[r, "f", bend left] \arrow[r, "g"', bend right] & B \arrow[l, "h"']
\end{tikzcd}
\]
where each bullet is replaced with the object the diagram maps it to and each arrow is labelled with the morphism the diagram maps it to. A diagram commutes iff the composites of the morphisms along any two path of arrows sharing starting and ending object are equal. For example, the LHS diagram above commutes iff $f=g\circ h$, the middle diagram above commutes iff $h\circ f=i\circ g$, and the RHS diagram commutes iff $f=g$ and $h\circ f=h\circ g=\mathbf 1_A$ and $f\circ h=g\circ h=\mathbf 1_B$. Note that the defining property of functors is that they preserve commutative diagrams: suppose $\mathscr{F}:\mathcal{I}\rightarrow\mathcal C$ is a commutative diagram and $\mathscr{G}:\mathcal C\rightarrow \mathcal D$ any functor, then $\mathscr G\circ \mathscr F$ is a commutative diagram.
\end{example}
\begin{example}Assuming some parts of a diagram commute, we can often show that other parts or the entire diagram commutes by exploiting properties of that diagram. For example, consider
\[
\begin{tikzcd}[sep = large]
A \arrow[d, "h"'] \arrow[r, "f"] & B \arrow[d, "i"] \arrow[r, "g"] & C \arrow[d, "j"] \\
D \arrow[r, "k"']                & E \arrow[r, "l"']               & F               
\end{tikzcd}
\]
Given that the two small squares commute, we can show that the large outer rectangle commutes. Since the two small squares commute, $i\circ f=k\circ h$ and $j\circ g=l\circ i$. Thus we have
\[j\circ g\circ f=l\circ i\circ f=l\circ k\circ h\]
Thus the outer rectangle commutes. Visually, we think of this process as ``chasing'' the morphisms.
\[
\begin{tikzcd}[sep = large]
A \arrow[d, "h"'] \arrow[r, "f", color=red] & B \arrow[d, "i"] \arrow[r, "g", color=red] & C \arrow[d, "j", color=red] \\
D \arrow[r, "k"']                & E \arrow[r, "l"']               & F 
\end{tikzcd}=
\begin{tikzcd}[sep = large]
A \arrow[d, "h"'] \arrow[r, "f", color=red] & B \arrow[d, "i",color=red] \arrow[r, "g"] & C \arrow[d, "j"] \\
D \arrow[r, "k"']                & E \arrow[r, "l"',color=red]               & F 
\end{tikzcd}=
\begin{tikzcd}[sep = large]
A \arrow[d, "h"', color=red] \arrow[r, "f"] & B \arrow[d, "i"] \arrow[r, "g"] & C \arrow[d, "j"] \\
D \arrow[r, "k"',color=red]                & E \arrow[r, "l"',color=red]               & F 
\end{tikzcd}
\]
This general technique is called \textit{diagram chasing}.
\end{example}
\begin{exercise}Consider the following diagram
\[\begin{tikzcd}
E \arrow[rrr] \arrow[ddd] \arrow[rd, "e"] &                       &                    & F \arrow[ddd] \arrow[ld] \\
                                          & A \arrow[r] \arrow[d] & B \arrow[d]        &                          \\
                                          & C \arrow[r]           & D \arrow[rd, "m"'] &                          \\
G \arrow[rrr] \arrow[ru]                  &                       &                    & H                       
\end{tikzcd}\]
Suppose that the four trapeziums commute, then show that
\begin{enumerate}[label=(\alph*)]
	\item if the inner square commutes then so does the outer square,
	\item if $e$ is epic, $m$ is monic, and the outer square commutes, then so does the inner square
\end{enumerate}
using diagram chasing.
\end{exercise}
\begin{definition}Let $\mathcal C,\mathcal D$ be categories, $\mathscr{F},\mathscr{G}:\mathcal{C}\rightarrow\mathcal D$ functors. A \textit{natural transformation} $\phi:\mathscr{F}\rightarrow\mathscr{G}$ is the data of its \textit{source} $\mathscr{F}$, \textit{target} $\mathscr{G}$, and a morphism $\phi(X):\mathscr{F}(X)\rightarrow\mathscr{G}(X)$ for each $X\in\mathcal C$ such that for all $X,Y\in\mathcal C$ and all morphisms $f:X\rightarrow Y$ the diagram
\[
\begin{tikzcd}[sep = huge]
\mathscr{F}(X) \arrow[r, "\phi(X)"] \arrow[d, "\mathscr{F}f"'] & \mathscr{G}(X) \arrow[d, "\mathscr{G}f"] \\
\mathscr{F}(Y) \arrow[r, "\phi(Y)"]                 & \mathscr{G}(Y)               
\end{tikzcd}\]
commutes, i.e. $\mathscr{G}f\circ\phi(X)=\phi(Y)\circ\mathscr{F}f$. The composition of natural transformations is the obvious element-wise composition. We define the \textit{functor category} from $\mathcal C$ to $\mathcal D$, denoted $\mathrm{Func}(\mathcal C,\mathcal D)$, as the category where objects are functors $\mathcal C\rightarrow \mathcal D$ and morphisms are natural transformations. For each category $\mathcal C$, we denote by $\mathrm{End}(\mathcal C)=\mathrm{Func}(\mathcal C,\mathcal C)$ whose objects are called \textit{endofunctors} on $\mathcal C$.
\end{definition}
\begin{example}Let $n\in\mathbf N$, consider functors $(-)^{\times},\mathrm{GL}_n(-):\mathbf{CRing}\rightarrow\mathbf{Grp}$ where $(-)^{\times}$ maps a ring $R$ to its group of units $R^\times$ and maps each ring homomorphism $f:R\rightarrow S$ to its restriction to the group of units $f^{\times}:R^\times\rightarrow S^\times$, and $\mathrm{GL}_n(-)$ maps a ring $R$ to the general linear group $\mathrm{GL}_n(R)$ and each ring homomorphism $f:R\rightarrow S$ to $\mathrm{GL}_n(f):\mathrm{GL}_n(R)\rightarrow \mathrm{GL}_n(S)$ by applying $f$ element-wise, i.e. $(r_{i,j})_{1\le i,j\le n}\mapsto (f(r_{i,j}))_{1\le i,j\le n}$. Define the natural transformation $\mathrm{det}:\mathrm{GL}_n(-)\rightarrow (-)^{\times}$ where for each ring $R$ the map $\mathrm{det}_R:\mathrm{GL}_n(R)\rightarrow R^{\times}$ is the determinant function. To see that this is well-defined, i.e. $f^\times \circ \mathrm{det}_{R}=\mathrm{det}_{S} \circ \mathrm{GL}_{n}(f)$ for each ring homomorphism $f:R\rightarrow S$, we note that the determinant function $\det_R$ maps $(r_{i,j})_{1\le i,j\le n}\mapsto \sum_{\sigma\in \mathrm{S}_n}\mathrm{sgn}(\sigma)\prod_{k=1}^nr_{k,\sigma(k)}$ which is given by the same formula regardless of the ring $R$.
\end{example}
\begin{example}Consider functor $\mathscr{C}:\mathbf{Ab}\rightarrow\mathbf{Ab}$ where for each abelian group $G$ the group $\mathscr{C}(G)$ is the group of functions of sets $g:\mathbf Z\rightarrow G$ under addition, and for each group homomorphism $f:G\rightarrow H$ the map $\mathscr{C}f:\mathscr{C}(G)\rightarrow\mathscr{C}(H)$ given by $g\mapsto f\circ g$ viewing $f$ as a function of sets. We define natural transformation $\Delta:\mathscr{C}\rightarrow\mathscr{C}$ where for each abelian group $G$ we have the morphism $\Delta_G:\mathscr{C}(G)\rightarrow \mathscr{C}(G)$ given by $\Delta_G(f)(n)=f(n+1)-f(n)$ for each $f\in\mathscr{C}(G)$ and $n\in\mathbf{Z}$, called the \textit{difference operator}. We note that for each morphism $f:G\rightarrow H$ in $\mathbf{Ab}$ we have $\mathscr{C}f\circ\Delta_G=\Delta_H\circ \mathscr{C}f$, since the difference operator is defined the same regardless of the abelian group $G$.
\end{example}

\begin{definition}An \textit{equivalence} of the categories $\mathcal C$ and $\mathcal D$ is a functor $\mathscr{F}:\mathcal C\rightarrow\mathcal D$ such that there exists a functor $\mathscr{G}:\mathcal D\rightarrow \mathcal{C}$ such that $\mathscr{F}\circ\mathscr{G}\cong \mathbf 1_{\mathcal D}$ in $\mathrm{End}(\mathcal D)$ and $\mathscr{G}\circ\mathscr{F}\cong \mathbf 1_{\mathcal C}$ in $\mathrm{End}(\mathcal C)$. If there exists an equivalence between $\mathcal C$ and $\mathcal D$, then we say $\mathcal C$ and $\mathcal D$ are \textit{equivalent} and write $\mathcal C\cong \mathcal D$.
\end{definition}
\begin{example}Suppose $G$ is a group, then $G$ can be viewed as a category $\mathbf BG$ called the \textit{delooping groupoid} of $G$, which is the category with one object $\bullet$ and all morphisms are the group elements $G$ with the group operation as composition. Fix $G$ and an algebraically closed field $k$. Recall that a representation of $G$ is a vector space $V\in\mathbf{Vect}_k$ and a homomorphism $\rho:G\rightarrow\mathrm{Aut}(V)$ where $\mathrm{Aut}(V)$ is the group of automorphisms on $V$. A morphism of representations $\phi:\langle V,\rho\rangle\rightarrow \langle W,\tau\rangle$ of $G$ is a linear map $\phi:V\rightarrow W$ such that $\phi\circ\rho(g)=\tau(g)\circ\phi$ for all $g\in G$, sometimes known as an \textit{equivariant map} or an \textit{intertwiner}. Straightforwardly, $\mathrm{Rep}(G)\cong \mathrm{Func}(\mathbf BG,\mathbf{Vect}_k)$ where $\mathrm{Rep}(G)$ is the category of representations of $G$. Moreover, $\mathrm{Rep}(G)\cong \mathbf{Mod}_{k[G]}$ where $k[G]$ is the group ring of $G$ over $k$, via the equivalence $\phi:\mathrm{Rep}(G)\rightarrow \mathbf{Mod}_{k[G]}$ where $\phi(\langle V,\rho\rangle)$ is the module with $V$ as its underlying set and action given by 
\[\left(\sum_{g \in G} \lambda_{g} g\right) v=\sum_{g \in G} \lambda_{g} \rho(g)(v)\]
It is left to the reader to verify that this is indeed an equivalence.
\end{example}

\begin{theorem}A functor is an equivalence iff it is fully faithful and essentially surjective.
\begin{proof}See appendix.
\end{proof}
\end{theorem}

\begin{theorem}[Yoneda lemma]\label{thm:yoneda}Let $\mathcal C$ be a locally small category. Let $A\in\mathcal C$ and let $\mathscr{F}:\mathcal C\rightarrow\mathbf{Set}$ be a functor, then there exists a canonical isomorphism
\[\mathrm{Mor}_{\mathrm{Func}(\mathcal C,\mathbf{Set})}(\mathscr{H}_A,\mathscr{F})\cong \mathscr{F}(A)\]
in $\mathbf{Set}$, given as follows: each natural transformation $\phi:\mathscr H_A\rightarrow\mathscr{F}$ is mapped to $\phi(A)(\mathbf 1_A)\in\mathscr{F}(A)$, and its inverse maps each $u\in\mathscr{F}(A)$ to the natural transformation $\phi:\mathscr H_A\rightarrow\mathscr{F}$ where for $X\in\mathcal C$ the function $\phi(X):\mathrm{Mor}_{\mathcal C}(A,X)\rightarrow \mathscr{F}(X)$ is given by $f\mapsto \mathscr{F}f(u)$.
\begin{proof}
For each natural transformation $\phi:\mathscr{H}_A\rightarrow\mathscr{F}$, if $f:A\rightarrow X$ is a morphism, then
\[
\begin{tikzcd}[sep = huge]
\mathrm{Mor}_{\mathcal C}(A,A) \arrow[r, "\phi(A)"] \arrow[d, "(f\circ -)"'] & \mathscr{F}(A) \arrow[d, "\mathscr{F}f"] \\
\mathrm{Mor}_{\mathcal C}(A,X) \arrow[r, "\phi(X)"]                 & \mathscr{F}(X)               
\end{tikzcd}\]
commutes. Thus $\phi(X)\circ (f\circ -)=\mathscr{F}f\circ \phi(A)$. Taking the value $\mathbf 1_A$ on both sides yields $\phi(X)(f)=\mathscr{F}f(\phi(A)(\mathbf 1_A))$, so $\phi$ is completely determined by $\phi(A)(\mathbf{1}_A)$. Conversely, for each $u\in\mathscr{F}(A)$, define the natural transformation $\phi:\mathscr H_A\rightarrow\mathscr{F}$ where for $X\in\mathcal C$ the function $\phi(X):\mathrm{Mor}_{\mathcal C}(A,X)\rightarrow \mathscr{F}(X)$ is given by $f\mapsto \mathscr{F}f(u)$, then we have $\phi(A)(\mathbf 1_A)=\mathscr{F}\mathbf 1_A(u)=\mathbf 1_{\mathscr{F}(A)}(u)=u$.
\end{proof}
\end{theorem}
\begin{corollary}[co-Yoneda lemma]Let $\mathcal C$ be a locally small category. Let $A\in\mathcal C$ and $\mathscr{F}:\mathcal C\rightarrow\mathbf{Set}$ a contravariant functor, viewing $\mathscr{H}^A,\mathscr{F}:\mathcal C^{\mathrm{op}}\rightarrow\mathbf{Set}$ as covariant functors, exists isomorphism
\[\mathrm{Mor}_{\mathrm{Func}(\mathcal C^{\mathrm{op}},\mathbf{Set})}(\mathscr{H}^A,\mathscr{F})\cong \mathscr{F}(A)\]
in $\mathbf{Set}$, given as follows: each natural transformation $\phi:\mathscr H^A\rightarrow\mathscr{F}$ is mapped to $\phi(A)(\mathbf 1_A)\in\mathscr{F}(A)$, and its inverse maps each $u\in\mathscr{F}(A)$ to the natural transformation $\phi:\mathscr H^A\rightarrow\mathscr{F}$ where for $X\in\mathcal C$ the function $\phi(X):\mathrm{Mor}_{\mathcal C^{\mathrm{op}}}(A,X)\rightarrow \mathscr{F}(X)$ is given by $f\mapsto \mathscr{F}f(u)$.
\begin{proof}Completely analogous to \autoref{thm:yoneda}
\end{proof}
\end{corollary}
\begin{corollary}\label{cor:yoneda}Let $\mathcal C$ be a locally small category. Let $A,B\in\mathcal C$, there exists $\mathbf{Set}$-isomorphisms
\begin{align*}
\mathrm{Mor}_{\mathrm{Func}(\mathcal C,\mathbf{Set})}(\mathscr{H}_A,\mathscr{H}_B)&\cong \mathscr{H}_B(A)=\mathrm{Mor}_{\mathcal C}(B,A)\tag{i}\\
\mathrm{Mor}_{\mathrm{Func}(\mathcal C^{\mathrm{op}},\mathbf{Set})}(\mathscr{H}^A,\mathscr{H}^B)&\cong \mathscr{H}^B(A)=\mathrm{Mor}_{\mathcal C}(A,B)\tag{ii}
\end{align*}
given by Yoneda lemma and co-Yoneda lemma. Next, if $\mathscr{H}_\bullet:\mathcal C\rightarrow\mathrm{Func}(\mathcal C,\mathbf{Set})$ is a contravariant functor where $\mathscr{H}_\bullet(A)=\mathscr{H}_A$ for $A\in\mathcal C$ and the natural transformation $\mathscr{H}_\bullet f$ for each $f:A\rightarrow B$ is given by (i), and $\mathscr{H}^\bullet:\mathcal C\rightarrow\mathrm{Func}(\mathcal C,\mathbf{Set})$ is the covariant functor where $\mathscr{H}^\bullet(A)=\mathscr{H}^A$ for $A\in\mathcal C$, and the natural transformation $\mathscr{H}^\bullet f$ for each $f:A\rightarrow B$ is given by (ii), then both $\mathscr{H}^\bullet$ and $\mathscr{H}_\bullet$ are fully faithful (and hence they are called the covariant and contravariant \textit{Yoneda embeddings}).
\begin{proof}Trivial.
\end{proof}
\end{corollary}
\begin{corollary}\label{cor:yoneda2}Let $\mathcal C$ be a locally small category, then
\[A\cong B \Longleftrightarrow\mathscr{H}^A\cong\mathscr{H}^B \Longleftrightarrow\mathscr{H}_A\cong\mathscr{H}_B\]
for all $A,B\in\mathcal C$.
\begin{proof}Follows from \autoref{cor:yoneda} and the fact that fully faithful functors are conservative.
\end{proof}
\end{corollary}
\begin{remark}Informally, \autoref{cor:yoneda2} tells us that each object is completely encoded by $\mathscr{H}^\bullet$ (resp. $\mathscr{H}_\bullet$), i.e. each object is determined by the collection of all its relationships with other objects, i.e. the ``in-arrows'' (resp. ``out-arrows''). Interestingly, there is a quote by philosopher Karl Marx about human nature reminiscent of this view, ``but the essence of man is no abstraction inherent in each single individual. In reality, it is the ensemble of the social relations.''
\end{remark}
\begin{remark}Informally, an isomorphism of the form $P(X)\cong Q(X)$ in a category is \textit{functorial} or \textit{natural} in the variable $X$ if both $P,Q$ can be viewed as functors in the variable $X$ in a natural way and the isomorphism viewed as a natural transformation is an isomorphism of functors, in which case the isomorphism is called a \textit{natural isomorphism}. For example, the Yoneda isomorphism $\mathrm{Mor}_{\mathrm{Func}(\mathcal C,\mathbf{Set})}(\mathscr{H}_A,\mathscr{F})\cong \mathscr{F}(A)$ is a natural isomorphism functorial in $A\in\mathcal C$ and $\mathscr{F}\in\mathrm{Func}(\mathcal C,\mathbf{Set})$.
\end{remark}
\begin{example}Let $\Delta$ be the \textit{simplex category}: the category where objects are all sets of the form $[n]=\{0,\dots,n\}$ for $n\in\mathbf N$, and morphisms are (non-strictly) increasing functions. A \textit{simplicial set} is a contravariant functor $\mathscr{X}:\Delta\rightarrow \mathbf{Set}$, and the category of simplicial sets $\mathbf{SSet}=\mathrm{Func}(\Delta^{\mathrm{op}},\mathbf{Set})$. For each $n\in\mathbf N$, the \textit{standard $n$-simplex} $\Delta^n=\mathscr{H}^{[n]}\in\mathbf{SSet}$. By co-Yoneda lemma, for each $\mathscr{X}\in \mathbf{SSet}$, we have $\mathrm{Mor}_{\mathbf{SSet}}(\Delta^n,\mathscr X)\cong \mathscr X_n$ where $\mathscr X_n=\mathscr X([n])$.
\end{example}
\begin{definition}Suppose $\mathcal C$ is a locally small category. A covariant (resp. contravariant) functor $\mathscr{F}:\mathcal C\rightarrow \mathbf{Set}$ is \textit{represented} by some $A\in\mathcal C$ if $\mathscr{F}\cong \mathscr{H}_A$ (resp. $\mathscr{F}\cong \mathscr{H}^A$), and it is \textit{representable} if it is represented by some $A\in\mathcal C$.
\end{definition}

\newpage
\section{Universal Property, Limits, and Adjoint Functors}

\section{Abelian Categories and Homological Algebra}
\section{Varieties}
Let $k$ be a field and $n\in\mathbf N$, then the \textit{affine space} $\mathbf A^n(k)$ of dimension $n$, or simply $\mathbf A^n$ if it does not cause confusion, is the same structure as the $n$-dimensional vector space $k^n$ over $k$, except with affine maps as morphisms, where an affine map is a linear map shifted by a constant. 

\begin{definition}
Suppose that $S\subseteq k[X_1,\dots,X_n]$ for some $n\in\mathbf N$, we define
\[\mathcal{V}(S)=\{x\in \mathbf A^n(k):\forall f\in S,\,f(x)=0\}\]
as the \textit{zero-locus} of $S$. A subset of $\mathbf A^n(k)$ that is the zero-locus of some $S$ is called \textit{(affine) algebraic}.
\end{definition}
For example, in $\mathbf{A}^2(\mathbf R)$, the sets $\mathcal{V}(\{Y\})$ and $\mathcal{V}(\{X\})$ are the $X$-axis and the $Y$-axis respectively, and the set $\mathcal{V}(\{X,Y\})$ is the origin. These are all examples of algebraic sets. As a non-example, the set $\{(\cos t,\sin t,t)\in\mathbf A^3(\mathbf R):t\in\mathbf R\}$ is not algebraic, as there is a line whose intersection with it is a infinite discrete set of points. Next, we claim that to find all algebraic sets, we need not consider all subsets of $k[X_1,\dots,X_n]$. Let $R$ be a commutative ring, we recall the following theorems.
\begin{theorem}$R$ is noetherian iff all ideals $I\subseteq R$ are finitely generated.
\end{theorem}
\begin{theorem}[Hilbert's Basis theorem]If $R$ is noetherian, then so is $R[X_1,\dots,X_n]$.
\end{theorem}

Therefore $k[X_1,\dots,X_n]$ is noetherian, and hence every for each $S\subseteq k[X_1,\dots,X_n]$, the ideal $\langle S\rangle$ is generated by some $f_1,\dots,f_m\in k[X_1,\dots,X_n]$. We claim that $\mathcal{V}(S)=\mathcal{V}(\langle S\rangle)=\mathcal{V}(\{f_1,\dots,f_m\})$. First, we know that $\mathcal{V}(\langle S\rangle)\subseteq \mathcal{V}(S)$ since $S\subseteq\langle S\rangle$. Conversely, suppose that $f\in\langle S\rangle$, then there exists $g_1,\dots,g_\ell\in S$ and $\lambda_1,\dots,\lambda_\ell\in k[X_1,\dots,X_n]$ with $f=\lambda_1g_1+\cdots+\lambda_{\ell}g_{\ell}$. Suppose that $p\in \mathcal{V}(S)$, then $f(p)=\lambda_1g_1(p)+\cdots+\lambda_{\ell}g_{\ell}(p)=0$. Since all $p\in \mathcal{V}(S)$ is a zero of all $f\in \langle S\rangle$, we must have $\mathcal{V}(S)\subseteq \mathcal{V}(\langle S\rangle)$, and hence $\mathcal{V}(S)=\mathcal{V}(\langle S\rangle)$. The equality $\mathcal{V}(\langle S\rangle)=\mathcal{V}(\{f_1,\dots,f_m\})$ is derived similarly. Therefore, it is a common abuse of notation to write $\mathcal{V}(f_1,\dots,f_n)$ instead of $\mathcal{V}(\{f_1,\dots,f_n\})$. Conversely, we can define an ideal $\mathcal{I}(X)$ for each $X\subseteq \mathbf A^n(k)$. 

\begin{definition}Let $X\subseteq \mathbf A^n(k)$, then define the ideal $\mathcal{I}(X)$ of $k[X_1,\dots,X_n]$ as 
\[\mathcal{I}(X)=\{f\in k[X_1,\dots,X_n]:\forall p\in X,\,f(p)=0\}\]
which is a well-defined ideal as we can verify easily.
\end{definition}
In fact, not only is $\mathcal{I}(X)$ an ideal, it is also a radical ideal. We recall that an radical ideal of a commutative ring $R$ is an ideal $I\subseteq R$ with $I=\sqrt{I}$ where $\sqrt{I}=\{r\in R:\exists m>0,\, r^m\in I\}$. In other words, a radical ideal is an ideal $I\subseteq R$ where for all $r\in R$, if $r^m\in I$ for some $m>0$, then $r\in I$. To see that $\mathcal{I}(X)$ is a radical ideal, note that if $f^m\in \mathcal{I}(X)$ for some $m>0$, then $f^m(p)=0$ for all $p\in X$, then $f(p)=0$ for all $p\in X$ as $k$ is a field, hence $f\in \mathcal{I}(X)$. 

Similar to how $\mathcal{V}(S)\subseteq \mathcal{V}(T)$ when $T\subseteq S\subseteq k[X_1,\dots,X_n]$, we easily have $\mathcal{I}(X)\subseteq \mathcal{I}(Y)$ when $Y\subseteq X\subseteq \mathbf A^n(k)$. Let $f\in S\subseteq k[X_1,\dots,X_n]$, then by definition $f$ vanishes on all of $\mathcal{V}(S)$, therefore $f\in \mathcal{I}(\mathcal{V}(S))$. Let $p\in X\subseteq \mathbf A^n(k)$, then by definition $p$ is a zero of all polynomials of $\mathcal{I}(X)$, therefore $p\in \mathcal{V}(\mathcal{I}(X))$. Hence we have $S\subseteq \mathcal{I}(\mathcal{V}(S))$ and $X\subseteq \mathcal{V}(\mathcal{I}(X))$. From these facts, we derive that $\mathcal{I}(X)= \mathcal{I}(\mathcal{V}(\mathcal{I}(X)))$ and $\mathcal{V}(S)= \mathcal{V}(\mathcal{I}(\mathcal V(S)))$. In fact, we have the following.
\begin{theorembox}[Hilbert's Nullstellensatz]There is a bijective correspondance
\[\{\textrm{radical\ ideals\ of\ }k[X_1,\dots,X_n]\}\longleftrightarrow\{\textrm{algebraic\ sets\ of\ }\mathbf A^n(k)\}\]
given by $I\mapsto \mathcal{V}(I)$ and $X\mapsto \mathcal{I}(X)$.
\end{theorembox}
whose proof we will delay until later in this note. This observation is central to algebraic geometry. 

We observe that for a nonempty family of ideals $I_{\alpha}\subseteq k[X_1,\dots,X_n]$ indexed by $\alpha$, we have $\mathcal{V}(\sum_{\alpha}I_{\alpha})=\mathcal{V}(\bigcup_{\alpha}I_{\alpha})=\bigcap_{\alpha}\mathcal{V}(I_{\alpha})$. This should be easy to verify, and it tells us that the arbitrary intersection of algebraic sets is algebraic. Next, we observe that for ideals $I,J\subseteq k[X_1,\dots,X_n]$, we have $\mathcal{V}(I\cap J)=\mathcal{V}(I\cdot J)=\mathcal{V}(I)\cup \mathcal{V}(J)$, where the set $I\cdot J=\{fg:f\in I,g\in J\}$. Suppose that $p\in \mathcal{V}(I)\cup \mathcal{V}(J)$, assume without loss of generality that $p\in \mathcal{V}(I)$. For all $f\in I\cap J$, we have $f\in I$, so $f(p)=0$, hence $p\in \mathcal{V}(I\cap J)$, thus $\mathcal{V}(I)\cup\mathcal{V}(J)\subseteq \mathcal{V}(I\cap J)$. On the other hand, for each $fg\in I\cdot J$, we have $(fg)(p)=f(p)g(p)=0$, thus we have $\mathcal{V}(I)\cup\mathcal{V}(J)\subseteq \mathcal{V}(I\cdot J)$. Conversely, if $p\not\in \mathcal{V}(I)\cup \mathcal{V}(J)$, then there exists $f\in \mathcal{V}(I)$ and $g\in \mathcal{V}(J)$ such that $f(p)\ne 0$ and $g(p)\ne 0$, and hence $(fg)(p)=f(p)g(p)\ne 0$ as $k$ is a field. Since we know that $fg\in I\cdot J$ and $fg\in I\cap J$, we have $p\not\in \mathcal{V}(I\cdot J)$ and $p\not\in \mathcal{V}(I\cap J)$. Thus $\mathcal{V}(I\cap J),\mathcal{V}(I\cdot J)\subseteq \mathcal{V}(I)\cup\mathcal{V}(J)$, and hence we completed the proof. Moreover, since $IJ\subseteq I\cap J$, we have $\mathcal{V}(I)\cup\mathcal{V}(J)=\mathcal{V}(I\cap J)\subseteq\mathcal{V}(IJ)$, and since $I\cdot J\subseteq IJ$, we have $\mathcal{V}(IJ)\subseteq\mathcal{V}(I\cdot J)=\mathcal{V}(I)\cup\mathcal{V}(J)$. Hence $\mathcal{V}(IJ)=\mathcal{V}(I)\cup\mathcal{V}(J)$ as well. From these facts, we conclude that the complements of algebraic sets form a topology of the affine space. This topology is known as the \textit{Zariski topology}, and the Zariski closed sets are just the algebraic sets.


\begin{definitionbox}An algebraic set $V$ is \textit{irreducible} if it cannot be written as $V=V_1\cup V_2$ where the algebraic sets $V_1,V_2\subset V$ properly, and such a set is called an \textit{(algebraic) variety}.
\end{definitionbox}
\begin{exercise}If $\mathcal{I}(X)=\mathcal{I}(Y)$ for algebraic sets $X,Y$, then $X=Y$.
\end{exercise}
\begin{lemma}An algebraic set $V$ is a variety iff $\mathcal{I}(V)$ is prime.
\begin{proof}Suppose that $\mathcal{I}(V)$ is not prime, that $fg\in \mathcal{I}(V)$ and $f,g\not\in \mathcal{I}(V)$. We claim that 
\[V=(V\cap \mathcal{V}(f))\cup (V\cap \mathcal{V}(g))\]
Let $p\in V$, then $(fg)(p)=f(p)g(p)=0$, thus $f(p)=0$ or $g(p)=0$ since $k$ is a field. Hence we have $p\in \mathcal{V}(f)$ or $p\in \mathcal{V}(g)$. Therefore $V\subseteq (V\cap \mathcal{V}(f))\cup (V\cap \mathcal{V}(g))$, the other direction $(V\cap \mathcal{V}(f))\cup (V\cap \mathcal{V}(g))\subseteq V$ is obvious. Since $f\not\in \mathcal{I}(V)$, exists $p\in V$ with $f(p)\ne 0$. Thus $p\not\in \mathcal{V}(f)$. Thus $V\ne V\cap \mathcal{V}(f)$. Similarly, $V\ne V\cap \mathcal{V}(g)$, so $V$ is reducible. Conversely, assume $V=V_1\cup V_2$ where $V_1,V_2\subset V$ properly. We have $\mathcal{I}(V)\subset \mathcal{I}(V_1),\mathcal{I}(V_2)$ properly. Choose $f\in \mathcal{I}(V_1)\setminus \mathcal{I}(V)$ and $g\in \mathcal{I}(V_2)\setminus \mathcal{I}(V)$. For $p\in V$, we have $p\in V_1$ or $p\in V_2$, thus $f(p)=0$ or $g(p)=0$, so $(fg)(p)=f(p)g(p)=0$. Hence $fg\in \mathcal{I}(V)$, so $\mathcal{I}(V)$ is not prime.
\end{proof}
\end{lemma}


Take, for example, the algebraic set $V=\mathcal{V}(f,g)\subseteq \mathbf A^3(\mathbf R)$ where $f(x,y,z)=x^2+y^2+z^2-4$ and $g(x,y,z)=y^2+z^2-1$. Then $V$ is the intersection of the sphere of radius $2$, and the cylinder of radius $1$. In fact, we have a decomposition of $V$
\[V=\mathcal{V}(x-\sqrt{3},y^2+z^2-1)\cup \mathcal{V}(x+\sqrt{3},y^2+z^2-1)\]
into algebraic varieties. This is easy to visualize and check that it is true. In fact, we can do even better. We will show that each algebaic set has a unique decomposition into algebraic varieties. Suppose that $R$ is a commutative ring, we recall the following theorem.
\begin{theorem}\label{thm:min}$R$ is noetherian iff every nonempty set of ideals has a maximal element.
\end{theorem}
\begin{theorem}If $V$ is an algebraic set, then $V$ has a unique decomposition $V=V_1\cup\cdots\cup V_m$, where $V_1,\dots,V_m$ are varieties such that no one of them is contained in another.
\begin{proof}
Suppose that $\mathcal{L}$ is the set of algebraic sets that do not admit a finite variety decomposition, we will show that $\mathcal{L}=\emptyset$. Suppose the contrary, then $\mathcal{L}$ has a minimal element $V$ w.r.t inclusion by \autoref{thm:min} on $\mathcal{I}[\mathcal{L}]$. Since $V\in A$, we have $V$ is reducible, hence $V=V_1\cup V_2$ with algebraics sets $V_1,V_2\subset V$ properly. Since $V$ is minimal, we must have $V_1,V_2\not\in \mathcal{L}$. Thus $V_1,V_2$ admit finite variety decompositions, contradiction. Next, we show the uniqueness. Let $V=V_1\cup\cdots\cup V_m=W_1\cup\dots\cup W_h$ be decompositions, then $V_i=(V_i\cap W_1)\cup\cdots\cup (V_i\cap W_h)$, which by the irreducibility of $V_i$, tells us that $V_i\subseteq W_{\sigma(i)}$ for some $\sigma(i)$. Similarly $W_j\subseteq V_{\delta(j)}$ for some $\delta(j)$. Thus $V_i\subseteq W_{\sigma(i)}\subseteq V_{\delta(\sigma(i))}$. However, $V_i\subseteq V_{\delta(\sigma(i))}$ implies that $V_i=V_{\delta(\sigma(i))}$, so $i=\delta(\sigma(i))$ and $V_i=W_{\sigma(i)}$. 
\end{proof}
\end{theorem}

\begin{proposition}Suppose that $k$ is algebraically closed and $F\in k[X,Y]$ is a nonconstant with decomposition $F=F_1^{n_1}\cdots F_r^{n_r}$ into irreducible polynomials, then $\mathcal{V}(F)=\mathcal{V}(F_1)\cup\cdots\cup \mathcal{V}(F_r)$ is the decomposition of $\mathcal{V}(F)$ into varieties, and $\mathcal{I}(\mathcal{V}(F))=(F_1\cdots F_r)$.
\end{proposition}

\section{Hilbert's Nullstellensatz}

\chapter{Schemes}
\section{Presheaves and Sheaves}
Let $X$ be a topological space, and $\mathcal C$ one of the categories $\mathbf{Set},\mathbf{Ab},\mathbf{CRing}$, or $\mathbf {Mod}_R$.
\begin{definition}\label{def:presheaf}A \textit{presheaf} $\mathscr{F}$ of $\mathcal C$ on $X$ is the data of the following
\begin{enumerate}[label=(\roman*)]
	\item for each open $U\subseteq X$, some $\mathscr{F}(U)\in \mathcal C$, whose elements are called the \textit{sections} of $\mathscr{F}$ over $U$,
	\item for each inclusion $U\hookrightarrow V$ of open sets of $X$, a \textit{restriction} map $\mathrm{Res}^V_U:\mathscr{F}(V)\rightarrow\mathscr{F}(U)$, s.t.
	\begin{itemize}
		\item for all open $U\subseteq X$, the map $\mathrm{Res}^U_U=\mathbf 1_{\mathscr{F}(U)}$,
		\item if $U\hookrightarrow V \hookrightarrow W$ are inclusions of open sets of $X$ then the diagram
		\[
		\begin{tikzcd}[sep=huge]
			\mathscr{F}(W) \arrow[r,"\mathrm{Res}^W_V"] \arrow[rr, "\mathrm{Res}^W_U", bend left, shift left=1] & \mathscr{F}(V) \arrow[r,"\mathrm{Res}^V_U"] & \mathscr{F}(U)
		\end{tikzcd}
		\]
		commutes, or in other words $\mathrm{Res}^W_U=\mathrm{Res}^V_U\circ \mathrm{Res}^W_V$
	\end{itemize}
\end{enumerate}
Moreover, suppose that $\mathscr{F},\mathscr{G}$ are presheaves of $\mathcal C$ on $X$, then a morphism $\phi:\mathscr{F}\rightarrow\mathscr{G}$ is the data of a map $\phi(U):\mathscr{F}(U)\rightarrow\mathscr{G}(U)$ for each open $U\subseteq X$ such that whenever $U\hookrightarrow V$ is an inclusion of open sets, the following diagram (where we abuse the notation $\mathrm{Res}^V_U$)
\[
\begin{tikzcd}[sep = huge]
\mathscr{F}(V) \arrow[r, "\phi(V)"] \arrow[d, "\mathrm{Res}^V_U"'] & \mathscr{G}(V) \arrow[d, "\mathrm{Res}^V_U"] \\
\mathscr{F}(U) \arrow[r, "\phi(U)"]                 & \mathscr{G}(U)               
\end{tikzcd}\]
commutes, i.e. $\mathrm{Res}^V_U\circ\phi(V)=\phi(U)\circ\mathrm{Res}^V_U$. The composition of these morphisms is element-wise.
\end{definition}
In fact, let $\mathcal{G}_X$ be the category of open sets of $X$, i.e. the objects of $\mathcal{G}_X$ are the open sets of $X$ and the morphisms are the inclusions of open sets, then a presheaf of $\mathcal C$ on $X$ is the same information as a contravariant functor from $\mathcal{G}_X$ to $\mathcal C$. Moreover, a morphism between presheaves of $\mathcal C$ on $X$ is the same information as a natural transformation between contravariant functors from $\mathcal{G}_X$ to $\mathcal C$. Thus the category of presheaves of $\mathcal C$ on $X$ is just the contravariant functor category $\mathrm{Func}^{\mathrm{op}}(\mathcal{G}_X,\mathcal C)$. One can check that this is just a restatement of \autoref{def:presheaf}. Therefore, we will view a presheaf of $\mathcal C$ on $X$ just as a contravariant functor from $\mathcal{G}_X$ to $\mathcal C$. 
\begin{definition}A presheaf $\mathscr{F}$ of $\mathcal C$ on $X$ is a \textit{sheaf} if for any open set $U\subseteq X$ and any open cover $\{U_i\}_{i\in\mathcal{I}}$ of $U$ the following two axioms are satisfied:
\begin{enumerate}[label=(\roman*)]
	\item \textit{identity axiom}: if $f,g\in\mathscr{F}(U)$, then $\mathrm{Res}^U_{U_i}(f)=\mathrm{Res}^U_{U_i}(g)$ for all $i\in\mathcal{I}$ implies that $f=g$, 
	\item \textit{gluability axiom}: if $f_i\in \mathscr{F}(U_i)$ for all $i\in\mathcal{I}$ is such that $\mathrm{Res}^{U_i}_{U_i\cap U_j}(f_i)=\mathrm{Res}^{U_j}_{U_i\cap U_j}(f_j)$ for all $i,j\in\mathcal{I}$, then there exists some $f\in\mathscr{F}(U)$ such that $\mathrm{Res}^U_{U_i}(f)=f_i$ for all $i\in\mathcal{I}$.
\end{enumerate}
Moreover, we will write a \textit{(pre)sheaf of sets} on $X$ to mean a (pre)sheaf of $\mathbf{Set}$ on $X$, and a \textit{(pre)sheaf of rings} on $X$ to mean a (pre)sheaf of $\mathbf{CRing}$ on $X$, and so on for all other possibilities for $\mathcal C$.
\end{definition}
\begin{example}A motivating example is the following. Suppose that $X=\mathbf R^n$, then define a sheaf of rings $\mathscr{F}$ on $X$ by letting $\mathscr{F}(U)=\{\mathrm{smooth\ functions\ }U\rightarrow \mathbf R\}$ for all open $U\subseteq X$ with the natural operations. For an inclusion of open sets $U\hookrightarrow V$ the restriction map $\mathrm{Res}^V_U$ is defined in the obvious way. We can easily check that $\mathscr{F}$ is well-defined. This sheaf is the \textit{sheaf of rings of smooth functions}.
\end{example}
\begin{definition}Let $\mathscr{F}$ be a presheaf of $\mathcal C$ on $X$, the \textit{stalk} $\mathscr{F}_p$ of $\mathscr{F}$ at a point $p\in X$, whose elements are called \textit{germs} of $\mathscr{F}$ at $p$, is an object of $\mathcal C$ where
\[\mathscr{F}_p=\left.\left\{(f,U):\begin{array}{c}f\in\mathscr{F}(U),\textrm{\ and\ }U\subseteq X\textrm{\ is}\\\textrm{an\ open\ neighborhood\ of\ }p\end{array}\right\} \middle/ \left\{\begin{array}{c}(f,U)\sim (g,V)\textrm{\ if\ and\ only\ if\ exists\ open}\\W\subseteq U\cap V\textrm{\ such\ that\ }\mathrm{Res}^U_W(f)=\mathrm{Res}^V_W(g)\end{array}\right\}\right.\]
with its extra structure inherited from $\mathscr{F}$ in the obvious way. For example, if $\mathcal C=\mathbf{CRing}$, then the extra structure of $\mathscr{F}_p$, the addition and multiplication, is defined as follows,
\[\begin{aligned} (f,U)+(g,V)&=(\mathrm{Res}^U_{U\cap V}(f)+\mathrm{Res}^V_{U\cap V}(g),U\cap V)\\ (f,U)(g,V)&=(\mathrm{Res}^U_{U\cap V}(f)\mathrm{Res}^V_{U\cap V}(g),U\cap V)\end{aligned}\]
We can check that this is well defined. In fact, we can formalize $\mathscr{F}_p$ as a colimit
\[\mathscr{F}_p=\lim_{\longrightarrow}\,\mathscr{F}(U)\textrm{\ indexed\ over\ }U\in\mathcal G_X\textrm{\ where\ }p\in U\]
where we are thinking of $\mathscr{F}$ as a diagram 
\end{definition}

\end{document}
